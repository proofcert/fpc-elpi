{\color{red}
The notion of \emph{focused proof systems}~\cite{andreoli92jlc,liang09tcs}
generalizes this view of proof construction by identifying the
following two phases. % of proof construction.
\begin{enumerate}
\item The \emph{negative}%
% \footnote{The terminology of negative and positive phases is a bit
%   unfortunate: historically, these terms do not refer to positive or negative
%   subformula occurrences but rather to certain semantic models used in
%   the study of linear logic~\cite{girard87tcs}.}
%
  phase corresponds to  goal-reduction: in this phase,
  inference rules that involve \emph{don't-care-non\-de\-ter\-min\-ism} are
  applied.
%
  As a result, there is no need to consider backtracking over choices
  made in building this phase.

\item The \emph{positive} phase corresponding to  backchaining: in this phase, inference rules that involve
  \emph{don't-know-non\-de\-ter\-min\-ism} are applied: here,
  inference rules need to be supplied with information in order to
  ensure that a completed proof can be built.
%
  That information can be items such as which term is needed to
  instantiate a universally quantified formula and which disjunct of a
  disjunctive goal formula should be proved.
\end{enumerate}
%
Thus, when building a proof tree (in the sequent calculus) from the
conclusion to its leaves, the negative phase corresponds to a simple
computation that needs no external information, while the positive
phase may need such external information to be supplied.
%
In the literature, it is common to refer to a repository of such
external information as either an \emph{oracle} or a \emph{proof
certificate}.
}

%%% Local Variables:
%%% mode: latex
%%% TeX-master: "main"
%%% End:
