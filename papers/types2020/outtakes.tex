
  % , and as is usual in Pure Type
% Systems we have the same grammar for terms and types
%
%, as shown in figure~\ref{fig:terms}.
% \todo{Let's think about that: we need A at least, but A is also a metavariable for atoms}
% %\begin{figure}
% \begin{align*}
% t,u ::= & k\ l \;|\; \forall x : t.\, u \\
% l ::= & [] \;|\; (t\ ::\ l)
% \end{align*}
% % \caption{Grammar of terms}
% % \label{fig:terms}
% % \end{figure}


% we denote by $\Pscr_A$ the
% set of type declarations for the constructors of type $A$, of the form
% $k : D$ where $D$ is of the form
% $\forall x_1 : t_1.\, \dots \forall x_n : t_n.\, D\ u_1 \dots u_m$. We
% denote by $\mathcal{A}$ the set of arity declaration for the inductive
% types.

%
%
%done Explain relations with other term assignments. Note that we do not use
% explicit substitutions (we are cut free)
% Explain the choice of not having certificates represent proof terms.
% \todo{  Explain that the unfold clause handles
% defined atoms and defined connectives (conjunction, disjunction, existential
% and also equality): this may be less efficient and proof-theoretically induce
% pointless phase changes, but it is arguably very elegant e simplifies proof
% terms that are inherited by Coq }




%% \begin{metanote}
%%   This paragraph seem to identify proof and type theory with sequents
%%   and nat ded and while I agree with the gist (although people such as
%%   Frank would argue that sequents are a way of organizing proof search
%%   for nat ded), we're not gonna make a lot of friends at types and
%%   more importantly is a big hammer to nail the point that logic
%%   programming is helpful with Coq --am
%% \end{metanote}

%% DM: I've tried to address your concerns by adding an additional
%% paragraph and references (below).


% DM I don't have a reference but I added more details.
%\begin{metanote}
%  `` partial fashion.'' Isn't that a bit vague? Some references? -am
%\end{metanote}


%% DM I no longer think this comment is useful for this paper, so I
%% delete it instead of trying to fix it.

%% Proof theory also has rather immediate and natural treatments for
%% co-induction (as well as induction: see
%% \cite{baelde12tocl,momigliano12jal}) whereas the treatment of
%% coinduction within type theory remains more challenging and is well
%% developed only for various syntactic conditions (see, for example,
%% \cite{bertot08entcs}).
%% \begin{metanote}
%%   The criticism applies to Coq, less to type theory, see co-patterns --am
%% \end{metanote}


%% The following paragraph seems out-of-place so I delete it now.

%% In this paper, we shall use two different proof systems.  The first
%% logic is first-order intuitionistic logic.  Here, predicates
%% are not inductively defined and quantification will be limited to
%% range over first-order term.  Actually, we shall examine only the Horn
%% clause fragment of that logic.
%% %
%% The second logic will allow for inductively defined predicates using a
%% fixed point notation.   Given again our restriction to essentially
%% Horn clause specifications, it is possible to view intuitionistic
%% reasoning with such specification as if they are actually
%% specification in $\mu$MALL, an extension of multiplicative additive
%% linear logic with the least and greatest fixed point operators
%% \cite{baelde12tocl,baelde07lpar,heath19jar}.


\begin{metanote}
AM:  simple are not poly: can we say ML-style polymorphism?
\end{metanote}

% \infer{\XXi\vdash t = t}
% {%\eqExpert{\XXi}
% }
% \qquad
% \infer{\XXi\vdash \true}
% {%\trueExpert{\XXi}
% }
% \qquad
% \infer{\XXi\vdash G_1\vee G_2}
% {\XXi\vdash G_i%\qquad \orExpert{\XXi}{\XXi'}{i}
% }
% \qquad
% \infer{\XXi\vdash \exists x. G}
% {\XXi\vdash G[t/x]%\qquad \someExpert{\XXi}{\XXi'}{t}
% }
% \qquad 
% \qquad
% \infer{\XXi;\Gamma\vdash \lambda x:A.\,t : \forall x : A.\,  G}
% {\XXi;\Gamma, x : A\vdash t : G%\quad \andExpert{\XXi}{\XXi_1}{\XXi_2}
% }
%\qquad


{\color{red}
The notion of \emph{focused proof systems}~\cite{andreoli92jlc,liang09tcs}
generalizes this view of proof construction by identifying the
following two phases. % of proof construction.
\begin{enumerate}
\item The \emph{negative}%
% \footnote{The terminology of negative and positive phases is a bit
%   unfortunate: historically, these terms do not refer to positive or negative
%   subformula occurrences but rather to certain semantic models used in
%   the study of linear logic~\cite{girard87tcs}.}
%
  phase corresponds to  goal-reduction: in this phase,
  inference rules that involve \emph{don't-care-non\-de\-ter\-min\-ism} are
  applied.
%
  As a result, there is no need to consider backtracking over choices
  made in building this phase.

\item The \emph{positive} phase corresponding to  backchaining: in this phase, inference rules that involve
  \emph{don't-know-non\-de\-ter\-min\-ism} are applied: here,
  inference rules need to be supplied with information in order to
  ensure that a completed proof can be built.
%
  That information can be items such as which term is needed to
  instantiate a universally quantified formula and which disjunct of a
  disjunctive goal formula should be proved.
\end{enumerate}
%
Thus, when building a proof tree (in the sequent calculus) from the
conclusion to its leaves, the negative phase corresponds to a simple
computation that needs no external information, while the positive
phase may need such external information to be supplied.
%
In the literature, it is common to refer to a repository of such
external information as either an \emph{oracle} or a \emph{proof
certificate}.
}

%%% Local Variables:
%%% mode: latex
%%% TeX-master: "main"
%%% End:
