\section{Elaboration of external proof certificates for the Coq kernel}
\label{sec:elab}

The trusted base of Coq is its kernel, which is a type-checking
program that certifies that a dependently typed $\lambda$-term has a
given type.  If type checking succeeds, the formula corresponding to
that type is, in fact, accepted by Coq users as a theorem of
intuitionistic logic (along with any axioms that have been asserted).
The rest of the Coq system, especially its tactic language, is
designed to help a human user build proofs-as-$\lambda$-terms that can
be checked by the kernel.

There are many theorem provers for intuitionistic logic
\cite{raths07jar} for which a successful proof is not the kind of
detailed $\lambda$-term required by the Coq kernel.  Often, such
provers provide no information about the proofs they discover.  To the
extent that some  evidence is output after a successful run, such
evidence is usually just a trace of some key aspects of a proof, where some
details are often not included.  For example:
\begin{enumerate}

\item Substitution instances of quantifiers might not be recorded in a
  proof since such instances can, in principle, be reconstructed using
  unification.

\item Detailed typing information might not need to be stored within a
  proof since types can often be reconstructed during proof checking.

\item Some simplifications steps might be applied within a proof
  without recording which rewrites were used.  A simple 
  non-deterministic proof-search engine might be expected to
  reconstruct an equivalent simplification.
\end{enumerate}
A majority of the external and automatic theorem provers for
intuitionistic logic do not involve induction.  Instead, they go
beyond Horn clause by permitting formulas with no restriction on
occurrences of $\forall$ and $\supset$.  In that case, we need to
modify the focused proof rules that we have seen in
Section~\ref{ssec:focused} by adding the following rules.
\[
  \infer[\supset R]{\Pscr\vdash B\supset D}{\Pscr, B\vdash D}
  \qquad
  \infer[\forall R]{\Pscr\vdash \forall x.D}{\Pscr\vdash D[y/x]}
\]
As usual, the $\forall R$ rule has the restriction that the
eigenvariable $y$ is not free in its conclusion.

As has been detailed in earlier work on foundational proof
certificates, this richer notion of proof system can provide for
richer proof certificates.  The main differences with what we have
seen before is that the left-hand sides of sequents can now grow
during the proof checking process.  When reading the right
introduction rule for $\supset$ from conclusion to premise, we shall
say that the antecedent of the implication is \emph{stored} in the
left side of the context.  When this store action occurs, an
\emph{index} is used by the store command to name that new, left-hand
formula occurrence.  In this extended situation, the decide expert
uses the index of an assumption in order to enter a focus phase of
inference.  A full proof checking kernel for first-order
intuitionistic logic has been given in \cite{chihani17jar} so we do
not reproduce it here.

\begin{figure}
\lstinputlisting[basicstyle=\ttfamily,language=lprolog,linerange={start-end}]{code/deb-fpc.sig}
\lstinputlisting[basicstyle=\ttfamily,language=lprolog,linerange={start-end}]{code/deb-fpc.mod}
\caption{The FPC definition of De Bruijn notation as proof evidence.}
\label{fig:debruijn}
\end{figure}

To give an example, consider using untyped $\lambda$-terms encoded
using De Bruijn's notation as proof certificates for the propositional
intuitionistic logic over just $\supset$.  The fact that such terms
can be used as proof certificates for such formulas (denoting simple
types) can be formally defined using the FPC description in
Figure~\ref{fig:debruijn}.
%
Using the constants provided in that figure, the untyped
$\lambda$-term $\lambda x (x (\lambda y (y (\lambda z (x (\lambda
u\,z))))))$ can be encoded as the following Elpi term of type
\lsti{deb}.
%
\begin{lstlisting}[basicstyle=\ttfamily,language=lprolog]
(lambda (apply 0 [lambda (apply 0 [lambda (apply 2 
                                                    [lambda (apply 1 [])])])]))
\end{lstlisting}
Using the constructor \lsti{lc} and \lsti{args}, terms in De Bruijn
syntax (terms of type \lsti{deb}) are incorporated into proof
certificates (terms of type \lsti{cert}) along with other integer
arguments that are needed to compute offsets to address bound
variables.  

We describe here briefly how to use the technology behind FPCs and
logic programming in order to provide a flexible approach to
connecting external provers of first-order intuitionistic logic to
Coq.  Following the general outline that has been described in
\cite{blanco17cade,blanco20coq}, we assume that the following steps are
taken.
\begin{enumerate}
\item Modify an external prover to output some form of proof evidence
  (proof certificate) for formulas it claims are theorems.

\item Develop a formal definition of the semantics behind such proof
  certificates using the FPC framework.  The FPC for De Bruijn
  expressions given in Figure~\ref{fig:debruijn} is an example of this
  step.

\item Check proofs by executing the logic programming checker that is
  parameterized by the particular FPC definition.
\end{enumerate}
As we have mentioned in Section~\ref{ssec:fpc}, the logic programming
setting allows parallel checking and synthesizing of a pair of
certificates.  That is, during the checking of one certificate, it is
possible to synthesize, for example, a fully detailed term that is
appropriate for handing to the Coq kernel.
If one is interested only in building Coq kernel proof structures, we
can bypass the use of an explicit pairing operation and build the
synthesis of such proof structures directly into the FPC proof
checker.  (We took exactly this step in Figure~\ref{fig:augmented}
where proof checking involved both proof certificates as well as
Coq-term structures.)  If one is interested in checking only one kind
of external proof structure, then the FPC for that structure could
also be built into the checker (via, say, partial evaluation of logic
programs \cite{lloyd91jlp}).

As an example, consider an external theorem prover for propositional
intuitionistic logic which returns proof structures as untyped
$\lambda$-terms using De Bruijn's notation.   Using the FPC provided
in Figure~\ref{fig:debruijn} and the proof certificate checker in the
file \verb+ljf-dep.mod+ of the repository
\url{https://github.com/proofcert/fpc-elpi/},  the De Bruijn term
display above can be elaborated into a proper proof for the following
Coq theorem. 

\begin{lstlisting}[basicstyle=\ttfamily,language=lprolog]
Theorem dneg_peirce_mid : forall P Q: Prop, (((((P -> Q) -> P) -> P) -> Q) -> Q).
\end{lstlisting}

We note that this proof certificate checker and Coq proof synthesizer
is rather compact, comprising less than 90 lines of Elpi code.

%%% Local Variables:
%%% mode: latex
%%% TeX-master: "main"
%%% End:

% LocalWords:  intuitionistic dneg peirce forall basicstyle lprolog
% LocalWords:  Elpi lc args ljf dep
