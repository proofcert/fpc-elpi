\section{Property-based testing for Coq}

Simplest setting since we are only sending back to Coq a term (a
counterexample), not a general proof.

\begin{metanote}
  dump from previous paper
\end{metanote}

When using a focused proof system for logic extended with fixed
points, such as  employed in Bedwyr~\cite{baelde07cade} and described
in~\cite{baelde12tocl,heath17linearity}, proofs of formulas such as
\[
  \exists x [(\tau(x)\wedge P(x)) \wedge \neg Q(x)]
  \tag{*}\label{eq:full}
\]
are a single \emph{bipole}: that is, when reading a proof bottom up, a
positive phase is followed on all its premises by a single negative
phase that completes the proof.%
\footnote{The reader familiar with focusing will understand that there
are two ``polarized'' conjunctions, written in linear logic as
$\otimes$ and $\with$ or in classical and intuitionistic logics as
$\wedgep$ and $\wedgen$, respectively.  In this paper, we use simply
$\wedge$ to denote the positive biased conjunction.}
%
In particular, the positive phase corresponds to the \emph{generation} phase
and the negative phase corresponds to the \emph{testing} phase.
%

Instead of giving a full focused proof system of a logic including
fixed points (since, as we will argue, that proof system will not, in
fact, be needed to account for PBT ), we offer the following analogy.
%
Suppose that we are given a finite search tree and we are asked to
prove that there is a secret located in one of the nodes of that
tree.
%
A proof that we have found that secret can be taken to be a
description of the path to that node from the root of the tree: that
path can be seen as the proof certificate for that claim.
%
On the other hand, a proof that no node contains the secret is a
rather different thing: here, one expects to use a blind and
exhaustive search (via, say, depth-first or breath-first search) and
that the result of that search never discovers the secret.
%
A proof of this fact requires \emph{no} external information: instead it
requires a lot of computation involved with exploring the tree until
exhaustion.
%
This follows the familiar pattern where the positive (generate) phase
requires external information while the negative (testing) phase
requires none.
%
Thus, a proof certificate for $(\ref{eq:full})$ is also a proof
certificate for 
\[
  \exists x [\tau(x)\wedge P(x)].
  \tag{**}\label{eq:short}
\]
Such a certificate would contain the witness (closed) term $t$ for the
existential quantifier and sufficient information to confirm that
$P(t)$ can be proved (a proof of a typing judgment such as $\tau(t)$
is usually trivial).
%
%In short\ednote{AM: a bit repetitious}, a proof certificate for
%$(\ref{eq:short})$ can play the part 
%of a proof certificate also for $(\ref{eq:full})$, if we also use
%negation-as-finite-failure  with $Q(t)$.
%  ****
Since a proof certificate for the existence-of-a-counterexample formula
$(\ref{eq:full})$ can be taken as a proof certificate of
$(\ref{eq:short})$, then we only need to consider proof certificates
for Horn clause programs.
%
%Since the logic of Horn clauses is rather simple,
We illustrate next what such proofs and proof certificates look like
for the rather simple logic of Horn clauses.


%%% Local Variables:
%%% mode: latex
%%% TeX-master: "main"
%%% End:
