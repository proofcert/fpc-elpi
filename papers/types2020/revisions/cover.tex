\documentclass{article}

\begin{document}
\title{Revisions made to the paper \\
  ``Two applications of logic programming to Coq''}
\maketitle{}

\section*{Main Changes}

From the implementation point of view, we have extended the kernel to
account for conversion: this allows us to do Prolog search and
therefore PBT on Inductive definitions that mix deduction and
computation. We have also implemented some optimizations to the kernel
itself, as briefly mentioned in the conclusion. We have experimented
with adding Hints to the kernel via Coq-Elpi's \texttt{Accumulate DB}
and while this is doable, it is not particularly useful for the
intended application, namely PBT. What we're working on is modifying
the elaboration tactic of Sec 5, which already sees the local context,
to handle Hints and Inductive. All the pieces are in place, however,  given the deadline, this
will not be part of this submission.

\section*{Reviewer 1}
\begin{verbatim}
An introduction to the topic, or at least a section explaining
 the application in the paper, would be necessary to open the
 paper to a larger audience.
\end{verbatim}

The reviewer is asking for more of an introduction about FPC.  We
have done two things to the paper.  First, we have added a
paragraph providing more of an explanation about FPC in Section
3.2.2.  Second, since the FPCs in the ``property-based testing''
part of his paper are rather simple, we illustrate their
operation on a few examples.  Those examples can be understood
fully from the little bit of Prolog code used to describe them.
The role of FPCs in Section 5 is more substantial but we are able
to treat them more as a blackbox: the elaboration aspect of FPCs
come from using parallel interpretation within logic programming:
these concepts have now been illustrated in Section 4.

\begin{verbatim}
page 5, Figure 1: the first inference rule (the decide inference
 rule) seems wrong. Indeed you are adopting Coq's syntx
 Ind[p](\Gamma_I := \Gamma_C) where p is a list of abstractions
 over so called uniform parameters ...
The alternative is that the parameters are explicitly quantified both in
\Gamma_i and \Gamma_C (the way the inductive type packet is represented
internally into Coq) and p becomes just a number that says how many of the
abstractions are parameters.
\end{verbatim}

We meant the latter alternative. This is now made explicit in the text.

\begin{verbatim}
It is mysterious why you choose to only use constructors: you
 never justify this restriction. There is also an hint that you
 can use the Hints that eauto uses, but this is not reflected in
 your code. How can it be?
\end{verbatim}
As anticipated before, it is immediate to add another ``decide'' rule
that picks up previously proven lemmas. That would be close to the
functionality of \texttt{Hint Resolve}, but does not literally uses
Coq's Hints facility. It is feasible that to write some glue
code to connect the two, but handling all the fairly sophisticated features of the Hint
mechanism is out of the scope of this paper.
\begin{verbatim}
you are re-implementing a better eauto. However, your
implementation only applies as lemmas constructors of an inductive type.
Note: I understand, because I know it in advance, that a reason could be
that you plan to use LP negation as failure in the later section on testing.
Therefore you need a close world specification. If that's the main reason,
it should be made explicit in the paper.
\end{verbatim}
If we have an inductive predicate $p$ and then prove a lemma about it,
this does not violate the CWA, since the lemma is true in the same
least model of $p$.  Operationally though, it could alter NF's
capability of finding \emph{finite} failure, since the latter is
implemented by the vanilla meta-interpreter, which is not FPC-driven. At this stage
we do not see a reason to have lemmata in PBT. We have mentioned this in the paper.

\begin{verbatim}
page 1: put some references for "that has not been given a
 satisfactory representation in type theory" and for "have
 difficulty treating linear negation and the multiplicative
 disjunction"
\end{verbatim}

We modified the claims in that paragraph and added two references
to help support those modified claims.

\section*{Reviewer 2}

\begin{verbatim}
1. The paper is centered on the idea that it is possible to use
 Coq-elpi to construct tactics following the ideas from FPC, but
 it lacks a proper evaluation of the tactics to show its
 benefits. It would make for a stronger story to have a thorough
 evaluation with its competitors (eauto for the first tactic and
 QuickChick for the second), even in restricted scenarios.
\end{verbatim}

We meant this paper to be a proof-of-concept rather than a system
paper. We are not in the position to carry out a ``thorough
evaluation'' with Coq's automation not with QuickChick, if not for the
high-level, qualitative remarks we have made in the paper.  We have
carried out in QuickChick the example in Section 4, both with a
functional and a relational encoding. The code is present in our
examples repo. We have added a few remarks in the paper but space
prevents us form explaining the features of QuickChick and their
peculiarity. [AM: add more if I do more about dependent
terms]. Finally, we may have misstated the ambitions of our Prolog
tactic, which does not mean to compete with eauto, but merely is the
building block of FPC-driven PBT of relational specifications. We have rephrased its characterization in the paper.


\begin{verbatim}
2. The code is not available. The cited
 https://github.com/proofcert/fpc-elpi returns 404, and the
 search for fpc-elpi failed, as well inspecting the repos from
 the proofcert organization.
\end{verbatim}

We are sorry for not having converted our repository from private
to public.  We have fixed this problem now.

\begin{verbatim}
1. The proof term shown as the result of calling the tactic to
 prove i1 is missing some terms, namely, the inequalities
 proofs. Are these also automatically obtained by the tactic?
\end{verbatim}

Well spotted: we had simply forgot the inequality guard in the third clause of
\texttt{insert}. The proof terms now reflects those evidence.

\end{document}

% LocalWords:  FPC FPCs blackbox Coq PBT QuickChick repo CWA NF's

%%% Local Variables:
%%% mode: latex
%%% TeX-master: t
%%% End:
% LocalWords:  lemmata
