\documentclass{article}

\begin{document}
\title{Revisions made to the paper \\
  ``Two applications of logic programming to Coq''}
\maketitle{}

\section*{Main Changes}

\begin{itemize}
\item ``auto'' with conversion
\item ``auto'' with Hints (?)
\item other \dots
\end{itemize}

\section*{Reviewer 1}
\begin{verbatim}
An introduction to the topic, or at least a section explaining
 the application in the paper, would be necessary to open the
 paper to a larger audience.
\end{verbatim}

The reviewer is asking for more of an introduction about FPC.  We
have done two things to the paper.  First, we have added a
paragraph providing more of an explanation about FPC in Section
3.2.2.  Second, since the FPCs in the ``property-based testing''
part of his paper are rather simple, we illustrate their
operation on a few examples.  Those examples can be understood
fully from the little bit of Prolog code used to describe them.
The role of FPCs in Section 5 is more substantial but we are able
to treat them more as a blackbox: the elaboration aspect of FPCs
come from using parallel interpretation within logic programming:
these concepts have now been illustrated in Section 4.

\begin{verbatim}
page 5, Figure 1: the first inference rule (the decide inference
 rule) seems wrong. Indeed you are adopting Coq's syntx
 Ind[p](\Gamma_I := \Gamma_C) where p is a list of abstractions
 over so called uniform parameters.  You seem to never
 instantiate them in the rule. Moverover, the premise "(head A :
 T) \in \Gamma_I" is also suspicious. I would expect that premise
 to compute a substitution \sigma for p, to be applied to D in
 both premises (e.g. D\sigma).
\end{verbatim}

comment

\begin{verbatim}
It is mysterious why you choose to only use constructors: you
 never justify this restriction. There is also an hint that you
 can use the Hints that eauto uses, but this is not reflected in
 your code. How can it be?
\end{verbatim}
comment

\begin{verbatim}
page 1: put some references for "that has not been given a
 satisfactory representation in type theory" and for "have
 difficulty treating linear negation and the multiplicative
 disjunction"
\end{verbatim}

We modified the claims in that paragraph and added two references
to help support those modified claims.

\section*{Reviewer 2}

\begin{verbatim}
1. The paper is centered on the idea that it is possible to use
 Coq-elpi to construct tactics following the ideas from FPC, but
 it lacks a proper evaluation of the tactics to show its
 benefits. It would make for a stronger story to have a thorough
 evaluation with its competitors (eauto for the first tactic and
 QuickChick for the second), even in restricted scenarios.
\end{verbatim}
comment

\begin{verbatim}
2. The code is not available. The cited
 https://github.com/proofcert/fpc-elpi returns 404, and the
 search for fpc-elpi failed, as well inspecting the repos from
 the proofcert organization.
\end{verbatim}

We are sorry for not having converted our repository from private
to public.  We have fixed this problem now.

\begin{verbatim}
1. The proof term shown as the result of calling the tactic to
 prove i1 is missing some terms, namely, the inequalities
 proofs. Are these also automatically obtained by the tactic?
\end{verbatim}

comment

\end{document}
