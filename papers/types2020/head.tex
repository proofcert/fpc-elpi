\usepackage{amssymb,amsmath}
\usepackage{proof}
\usepackage{xspace}
\usepackage{listings}
\usepackage{letltxmacro}
\usepackage[dvipsnames]{xcolor}
\usepackage{cmll}  % Improved linear logic symbols

\definecolor{dkgreen}{RGB}{34,139,34}

% \newcommand{\blue}[1]{{\color[rgb]{0,0,1} #1}}
% \newcommand{\tup}[1]{\langle #1\rangle}
% \newcommand{\tupp}[2]{\blue{\langle #1,}#2{\blue{\rangle}}}

\newcommand{\instan}[1]{\hbox{\sl grnd}~(#1)}
\newcommand{\Pscr}{{\mathcal P}}

\newcommand{\ok}{\checkmark}
% DM The use of pifont and txfonts broke typesetting for me: equality
% signs never appearred and some parentheses did not appear either.
%\newcommand{\noc}{\ding{56}}
\newcommand{\noc}{\ddag}
\newcommand{\lP}{$\lambda$Prolog\xspace}
\newcommand{\nat }{\hbox{\sl nat}\xspace}
\newcommand{\plus}{\hbox{\sl app}\xspace}
\newcommand{\lst}{\hbox{\sl lst}\xspace}
\newcommand{\sort}{\hbox{\sl sort}}
\newcommand{\rev}{\hbox{\sl rev}}
\newcommand{\ordered}{\hbox{\sl ordered}}

%\newcommand{\atac}{A2Tac\xspace}
\newcommand{\atac}{ACheck\xspace}
%\newcommand{\lra}{\mathrel{\longrightarrow}}
\newcommand{\lra}{\mathrel{\vdash}}
\newcommand{\seq}[2]{#1\lra #2}

\newcommand{\true}{tt}

%%%%%%%%%%%%%%%% LJF
\newcommand{\truen}{t^-}
\newcommand{\truep}{t^+}
\newcommand{\falsen}{f^-}
\newcommand{\falsep}{f^+}
\newcommand{\wedgep}{\wedge^{\!+}}
\newcommand{\wedgen}{\wedge^{\!-}}
\newcommand{\veep}{\vee^{\!+}}
\newcommand{\veen}{\vee^{\!-}}
%\newcommand{\with}{\&}

\newcommand{\Nscr}{{\cal N}}
\newcommand{\Rscr}{{\cal R}}                                   % Used for an ambiguous rhs
%\newcommand{\Rscr}{\Delta_1\Downarrow\Delta_2}                % Used for an ambiguous rhs
\newcommand{\jUnf    }[4]{#1\mathbin\Uparrow#2\vdash#3\mathbin\Uparrow #4} % unfocused sequent
\newcommand{\jUnfG   }[2]{\jUnf{\Gamma}{#1}{#2}{{}}}           % unf sequ with \Gamma
\newcommand{\jUnfamb }[3]{#1\mathbin\Uparrow#2\vdash#3 \Rscr}  % unfocused sequent
\newcommand{\jUnfGamb}[1]{\Gamma\mathbin\Uparrow#1\vdash \Rscr}% unf sequ with \Gamma
\newcommand{\jLf     }[3]{#1\Downarrow#2\vdash#3}              % left focused sequent
\newcommand{\jLfG    }[1]{\jLf{\Gamma}{#1}{E}}                 % left foc seq with \Gamma 
\newcommand{\jRf     }[2]{#1\vdash #2\Downarrow}               % right focused sequent
\newcommand{\jRfG    }[1]{\jRf{\Gamma}{#1}}                    % right foc seq with \Gamma
%%%%%%%%%%%%%%%% 
\newcommand{\bxi}[1]{\blue{\Xi_{#1}}}
\newcommand{\bXi}[1]{\blue{\Xi_{#1} :\null}}

\newcommand{\andClerk}[3]{{\wedge_c}(#1,#2,#3)}
\newcommand{\falseClerk}[2]{f_c(#1,#2)}
\newcommand{\orClerk}[2]{{\vee_c}(#1,#2)}
\newcommand{\allClerk}[2]{\forall_c(#1,#2)}
\newcommand{\storeClerk}[3]{\hbox{\sl store}_c(#1,#2,#3)}

\newcommand{\trueExpert }[1]{{\true_e}(#1)}
\newcommand{\eqExpert }[1]{{=_e}(#1)}
\newcommand{\unfoldExpert}[2]{{\hbox{\sl unfold}_e}(#1,#2)}
\newcommand{\andExpert}[3]{{\wedge_e}(#1,#2,#3)}
\newcommand{\prodExpert}[3]{{\forall_e}(#1,#2,#3)}
\newcommand{\andExpertLJF}[6]{{\wedge_e}(#1,#2,#3,#4,#5,#6)}
\newcommand{\orExpert  }[3]{{\vee_e}(#1,#2,#3)}
\newcommand{\someExpert}[3]{\exists_e(#1,#2,#3)}
\newcommand{\initExpert}[2]{\hbox{\sl init}_e(#1,#2)}
\newcommand{\cutExpert}[4]{\hbox{\sl cut}_e(#1,#2,#3,#4)}
\newcommand{\decideExpert}[3]{\hbox{\sl decide}_e(#1,#2,#3)}
\newcommand{\releaseExpert}[2]{\hbox{\sl release}_e(#1,#2)}

\newcommand{\initLExpert}[1]{\hbox{\sl initial}_e(#1)}

%%%%%%%%%%%%%%%%

%
\newcommand{\step}{\longrightarrow}
\newcommand{\sstlc}{\textit{STLC}\xspace}
\newcommand{\vds}{\vdash_\Sigma}


\def\bnfas{\mathrel{::=}}
\def\bnfalt{\mid}

\def\lam{\lambda}
\def\Lam{\Lambda}
\def\arrow{\rightarrow}
\def\oftp{\mathord{:}}
\def\hastype{\mathrel{:}}
%%%%%%%%%%%%%%%%%%%%%%%%%%%%%%%%%%%%%%%%%%%%%%%%%%%%%%%%%%%%%%%%%%%%%%
% Editorials
%%%%%%%%%%%%%%%%%%%%%%%%%%%%%%%%%%%%%%%%%%%%%%%%%%%%%%%%%%%%%%%%%%%%%%

\long\def\ednote#1{\footnote{[{\it #1\/}]}\message{ednote!}}
%\long\def\ednote#1{\begin{quote}[{\it #1\/}]\end{quote}\message{note!}}
\newenvironment{metanote}{\begin{quote}\message{note!}[\begingroup\it}%
                         {\endgroup]\end{quote}}
\long\def\ignore#1{}

\newcommand{\todo}[1]{\begin{metanote}TODO: #1\end{metanote}}
%
%\setlength{\textwidth}{13.2cm}
%\setlength{\textheight}{19.9cm}


\input listing-macros  % Use to get better highlighing of code
\lstset{language=lprolog}
\lstset{language=abella}
\lstset{basicstyle = \small}
% lstlisting coq style (inspired from a file of Assia Mahboubi)
%
\lstdefinelanguage{Coq}{ 
%
% Anything betweeen $ becomes LaTeX math mode
mathescape=true,
%
% Comments may or not include Latex commands
texcl=false, 
%
% Vernacular commands
morekeywords=[1]{Section, Module, End, Require, Import, Export,
  Variable, Variables, Parameter, Parameters, Axiom, Hypothesis,
  Hypotheses, Notation, Local, Tactic, Reserved, Scope, Open, Close,
  Bind, Delimit, Definition, Let, Ltac, Fixpoint, CoFixpoint, Add,
  Morphism, Relation, Implicit, Arguments, Unset, Contextual,
  Strict, Prenex, Implicits, Inductive, CoInductive, Record,
  Structure, Canonical, Coercion, Context, Class, Global, Instance,
  Program, Infix, Theorem, Lemma, Corollary, Proposition, Fact,
  Remark, Example, Proof, Goal, Save, Qed, Defined, Hint, Resolve,
  Rewrite, View, Search, Show, Print, Printing, All, Eval, Check,
  Projections, inside, outside, Def},
%
% Gallina
morekeywords=[2]{forall, exists, exists2, fun, fix, cofix, struct,
  match, with, end, as, in, return, let, if, is, then, else, for, of,
  nosimpl, when},
%
% Sorts
morekeywords=[3]{Type, Prop, Set, true, false, option},
%
% Various tactics, some are std Coq subsumed by ssr, for the manual purpose
morekeywords=[4]{pose, set, move, case, elim, apply, clear, hnf,
  intro, intros, generalize, rename, pattern, after, destruct,
  induction, using, refine, inversion, injection, rewrite, congr,
  unlock, compute, ring, field, fourier, replace, fold, unfold,
  change, cutrewrite, simpl, have, suff, wlog, suffices, without,
  loss, nat_norm, assert, cut, trivial, revert, bool_congr, nat_congr,
  symmetry, transitivity, auto, split, left, right, autorewrite},
%
% Terminators
morekeywords=[5]{by, done, exact, reflexivity, tauto, romega, omega,
  assumption, solve, contradiction, discriminate},
%
% Control
morekeywords=[6]{do, last, first, try, idtac, repeat},
%
% Comments delimiters, we do turn this off for the manual
morecomment=[s]{(*}{*)},
%
% Spaces are not displayed as a special character
showstringspaces=false,
%
% String delimiters
morestring=[b]",
morestring=[d]’,
%
% Size of tabulations
tabsize=3,
%
% Enables ASCII chars 128 to 255
extendedchars=false,
%
% Case sensitivity
sensitive=true,
%
% Automatic breaking of long lines
breaklines=false,
%
% Default style fors listings
basicstyle=\small,
%
% Position of captions is bottom
captionpos=b,
%
% flexible columns
columns=[l]flexible,
%
% Style for (listings') identifiers
 identifierstyle={\ttfamily\tt},
% % Style for declaration keywords
keywordstyle=[1]{\ttfamily\bf},
% % Style for gallina keywords
keywordstyle=[2]{\ttfamily\bf},
% % Style for sorts keywords
 keywordstyle=[3]{\ttfamily\bf},
% % Style for tactics keywords
 keywordstyle=[4]{\ttfamily\bf},
% % Style for terminators keywords
 keywordstyle=[5]{\ttfamily\bf},
% %Style for iterators
keywordstyle=[6]{\ttfamily\color{dkpink}},
% % Style for strings
 stringstyle=\ttfamily,
% % Style for comments
 commentstyle={\ttfamily\color{dkgreen}},
% %
moredelim=**[is][\ttfamily\color{red}]{/&}{&/},
literate=
{forall}{{{$\forall\;$}}}1
{exists}{{$\exists\;$}}1
    {\\exists}{{$\exists\;$}}1
    {<-}{{$\leftarrow\;$}}1
    {=>}{{$\Rightarrow\;$}}1
    {==}{{\code{==}\;}}1
    {==>}{{\code{==>}\;}}1
%    {:>}{{\code{:>}\;}}1
    {->}{{$\rightarrow\;$}}1
    {<->}{{$\leftrightarrow\;$}}1
    {<==}{{$\leq\;$}}1
    {\#}{{$^\star$}}1 
    {\\o}{{$\circ\;$}}1 
%    {\@}{{$\cdot$}}1 
    {\/\\}{{$\wedge\;$}}1
    {\\\/}{{$\vee\;$}}1
    {++}{{\code{++}}}1
    {~}{{$\lnot$}}1
    {\@\@}{{$@$}}1
    {\\mapsto}{{$\mapsto\;$}}1
    {\\hline}{{\rule{\linewidth}{0.5pt}}}1
%
}[keywords,comments,strings]

\lstnewenvironment{coq}{\lstset{language=Coq}, basicstyle=\small}{}

% pour inliner dans le texte
\def\coqe{\lstinline[language=Coq, basicstyle=\small]}
\DeclareRobustCommand\lsti[1][]{\lstinline[basicstyle=\ttfamily,keepspaces=true,#1]}
% pour inliner dans les tableaux / displaymath...
\def\coqes{\lstinline[language=Coq, basicstyle=\small]}


%  LocalWords:  mathescape texcl morekeywords Ltac Fixpoint
%  LocalWords:  CoFixpoint

\lstset{language=coq}
\lstset{basicstyle=\small,
		breaklines=true}

\newcommand{\lpar}{\mathrel{\parr}}
\newcommand{\XXi}{{\color{blue}{\Xi}}}
\newcommand{\bc}[5]{#1#2\Downarrow #3 : #4 \vdash #5}
\newcommand{\depprod}[3]{(#1\colon#2)#3}
\newcommand{\prodE}[4]{\hbox{\sl prod}_e(#1,#2,#3,#4)}


% LocalWords:  grnd nat lst ACheck tt
