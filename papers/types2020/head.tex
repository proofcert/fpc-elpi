\usepackage{amssymb,amsmath}
\usepackage{proof}
\usepackage{xspace}
\usepackage{listings}
\usepackage{letltxmacro}
\usepackage[dvipsnames]{xcolor}
\usepackage{cmll}  % Improved linear logic symbols

\definecolor{dkgreen}{RGB}{34,139,34}

% \newcommand{\blue}[1]{{\color[rgb]{0,0,1} #1}}
% \newcommand{\tup}[1]{\langle #1\rangle}
% \newcommand{\tupp}[2]{\blue{\langle #1,}#2{\blue{\rangle}}}

\newcommand{\instan}[1]{\hbox{\sl grnd}~(#1)}
\newcommand{\Pscr}{{\mathcal P}}

\newcommand{\ok}{\checkmark}
% DM The use of pifont and txfonts broke typesetting for me: equality
% signs never appearred and some parentheses did not appear either.
%\newcommand{\noc}{\ding{56}}
\newcommand{\noc}{\ddag}
\newcommand{\lP}{$\lambda$Prolog\xspace}
\newcommand{\nat }{\hbox{\sl nat}\xspace}
\newcommand{\plus}{\hbox{\sl app}\xspace}
\newcommand{\lst}{\hbox{\sl lst}\xspace}
\newcommand{\sort}{\hbox{\sl sort}}
\newcommand{\rev}{\hbox{\sl rev}}
\newcommand{\ordered}{\hbox{\sl ordered}}

%\newcommand{\atac}{A2Tac\xspace}
\newcommand{\atac}{ACheck\xspace}
%\newcommand{\lra}{\mathrel{\longrightarrow}}
\newcommand{\lra}{\mathrel{\vdash}}
\newcommand{\seq}[2]{#1\lra #2}

\newcommand{\true}{tt}

%%%%%%%%%%%%%%%% LJF
\newcommand{\truen}{t^-}
\newcommand{\truep}{t^+}
\newcommand{\falsen}{f^-}
\newcommand{\falsep}{f^+}
\newcommand{\wedgep}{\wedge^{\!+}}
\newcommand{\wedgen}{\wedge^{\!-}}
\newcommand{\veep}{\vee^{\!+}}
\newcommand{\veen}{\vee^{\!-}}
%\newcommand{\with}{\&}

\newcommand{\Nscr}{{\cal N}}
\newcommand{\Rscr}{{\cal R}}                                   % Used for an ambiguous rhs
%\newcommand{\Rscr}{\Delta_1\Downarrow\Delta_2}                % Used for an ambiguous rhs
\newcommand{\jUnf    }[4]{#1\mathbin\Uparrow#2\vdash#3\mathbin\Uparrow #4} % unfocused sequent
\newcommand{\jUnfG   }[2]{\jUnf{\Gamma}{#1}{#2}{{}}}           % unf sequ with \Gamma
\newcommand{\jUnfamb }[3]{#1\mathbin\Uparrow#2\vdash#3 \Rscr}  % unfocused sequent
\newcommand{\jUnfGamb}[1]{\Gamma\mathbin\Uparrow#1\vdash \Rscr}% unf sequ with \Gamma
\newcommand{\jLf     }[3]{#1\Downarrow#2\vdash#3}              % left focused sequent
\newcommand{\jLfG    }[1]{\jLf{\Gamma}{#1}{E}}                 % left foc seq with \Gamma 
\newcommand{\jRf     }[2]{#1\vdash #2\Downarrow}               % right focused sequent
\newcommand{\jRfG    }[1]{\jRf{\Gamma}{#1}}                    % right foc seq with \Gamma
%%%%%%%%%%%%%%%% 
\newcommand{\bxi}[1]{\blue{\Xi_{#1}}}
\newcommand{\bXi}[1]{\blue{\Xi_{#1} :\null}}

\newcommand{\andClerk}[3]{{\wedge_c}(#1,#2,#3)}
\newcommand{\falseClerk}[2]{f_c(#1,#2)}
\newcommand{\orClerk}[2]{{\vee_c}(#1,#2)}
\newcommand{\allClerk}[2]{\forall_c(#1,#2)}
\newcommand{\storeClerk}[3]{\hbox{\sl store}_c(#1,#2,#3)}

\newcommand{\trueExpert }[1]{{\true_e}(#1)}
\newcommand{\eqExpert }[1]{{=_e}(#1)}
\newcommand{\unfoldExpert}[2]{{\hbox{\sl unfold}_e}(#1,#2)}
\newcommand{\andExpert}[3]{{\wedge_e}(#1,#2,#3)}
\newcommand{\prodExpert}[3]{{\forall_e}(#1,#2,#3)}
\newcommand{\andExpertLJF}[6]{{\wedge_e}(#1,#2,#3,#4,#5,#6)}
\newcommand{\orExpert  }[3]{{\vee_e}(#1,#2,#3)}
\newcommand{\someExpert}[3]{\exists_e(#1,#2,#3)}
\newcommand{\initExpert}[2]{\hbox{\sl init}_e(#1,#2)}
\newcommand{\cutExpert}[4]{\hbox{\sl cut}_e(#1,#2,#3,#4)}
\newcommand{\decideExpert}[3]{\hbox{\sl decide}_e(#1,#2,#3)}
\newcommand{\releaseExpert}[2]{\hbox{\sl release}_e(#1,#2)}

\newcommand{\initLExpert}[1]{\hbox{\sl initial}_e(#1)}

%%%%%%%%%%%%%%%%

%
\newcommand{\step}{\longrightarrow}
\newcommand{\sstlc}{\textit{STLC}\xspace}
\newcommand{\vds}{\vdash_\Sigma}


\def\bnfas{\mathrel{::=}}
\def\bnfalt{\mid}

\def\lam{\lambda}
\def\Lam{\Lambda}
\def\arrow{\rightarrow}
\def\oftp{\mathord{:}}
\def\hastype{\mathrel{:}}
%%%%%%%%%%%%%%%%%%%%%%%%%%%%%%%%%%%%%%%%%%%%%%%%%%%%%%%%%%%%%%%%%%%%%%
% Editorials
%%%%%%%%%%%%%%%%%%%%%%%%%%%%%%%%%%%%%%%%%%%%%%%%%%%%%%%%%%%%%%%%%%%%%%

\long\def\ednote#1{\footnote{[{\it #1\/}]}\message{ednote!}}
%\long\def\ednote#1{\begin{quote}[{\it #1\/}]\end{quote}\message{note!}}
\newenvironment{metanote}{\begin{quote}\message{note!}[\begingroup\it}%
                         {\endgroup]\end{quote}}
\long\def\ignore#1{}

\newcommand{\todo}[1]{\begin{metanote}TODO: #1\end{metanote}}
%
%\setlength{\textwidth}{13.2cm}
%\setlength{\textheight}{19.9cm}


\input listing-macros  % Use to get better highlighing of code
\lstset{language=lprolog}
\lstset{language=abella}
\input{lstcoq.sty}
\lstset{language=coq}
\lstset{basicstyle=\footnotesize,
		breaklines=true}

\newcommand{\lpar}{\mathrel{\parr}}
\newcommand{\XXi}{{\color{blue}{\Xi}}}
\newcommand{\bc}[5]{#1#2\Downarrow #3 : #4 \vdash #5}
\newcommand{\depprod}[3]{(#1\colon#2)#3}
\newcommand{\prodE}[4]{\hbox{\sl prod}_e(#1,#2,#3,#4)}


% LocalWords:  grnd nat lst ACheck tt
