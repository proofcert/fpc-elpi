\section{Introduction}
\label{sec:intro}

Recently, C. Sacerdoti Coen and E. Tassi developed the ELPI
implementation \cite{dunchev15lpar} of \lP \cite{miller12proghol}.
More recently, Tassi has made ELPI available as a plugin to the Coq
proof assistant \cite{tassi18coqpl}.  This implementation of \lP
extends earlier implementations in primarily two directions.  First,
ELPI adds a notion of constraints, which makes it more expressive than,
say, the Teyjus implementation of \lP \cite{nadathur99cade}.  Second,
the plugin version of ELPI comes equip with an API for accessing the
Coq environment and a quotation and anti-quotation syntax allowing for
Coq expressions to be mixed with \lP program elements.

ELPI provides a means for treating bindings and unification (of
structures containing bindings) that are convenient for various
meta-programming tasks in logic programming all things useful for
tactic programming in Coq (proof refinement).  For examples, ELPI
programmers are spared from dealing with low-level aspects of
representing bindings (e.g., De Bruijn indexes) while still having
clean and effective ways to manipulate binding structures.  In short,
ELPI appears to be a useful meta-language for Coq
\cite{coen19mscs,tassi18coqpl,tassi19itp}.  
Type inference is a well-known example of a meta-programming task for
which logic programming often provides immediate and elegant
implementations via the sound implementation of logic.  % In this paper,
% we present two additional project in which an implementation of proof
% search provides direct, elegant, and compact implementations of
% meta-programming tasks.
In this paper, we outline
two applications of the Coq plugin for ELPI that support this claim.
With these examples, we shall illustrate that ELPI is a useful
meta-language not just because for its treatment of bindings and
syntactic structures in general but also because it is actually based
on a sound implementation of a higher-order intuitionistic logic.
Before we present these two examples, we
first discuss the different world views that the integration of ELPI
and Coq forces us to confront.
