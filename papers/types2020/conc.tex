\section{Conclusion and future work}

\todo{sum up contribution}

% \begin{metanote}
%   Decided against a related work section: most is already discussed in the intro, the rest can be mentioned in passing --am
% \end{metanote}
% \subsection{Related work}
% \label{ssec:rel}

% \begin{itemize}
% \item Brief discussion/comparison with \texttt{auto}
% \item Mention other papers by Enrico as applications of coq-elpi~\cite{coen19mscs,tassi19itp}\dots [done]
% \item Say we \textbf{do} not talk about Coq metaprogramming (Ltac, Mtac), do say that doing our to application in MetaCoq~\cite{sozeau2020metacoq} would be a disaster (reimplement unfication, backtracking, binders) [done]
% \end{itemize}

% \subsection{Future work}
% \label{ssec:fut}
\todo{finish future work}
There are many avenues of developments for thisline of research.
The code reported in Fig.~\ref{fig:augmented} is somewhat idealized in
so far that we omitted a rule performing weak-head reduction (allowing
to handle directly existential goals) and more importantly that used
the \emph{delay} mechanism. This a distinguished feature of Elpi, whose
usefulness has been advocated in ~\cite{DunchevCT16,coen19mscs}.  The
use of constraints for data generation is well
developed~\cite{FioravantiPS15} and we could try to leverage to
improve PBT using partially instantiated terms: for example generating
say lists with at least two elements as \texttt{[1,2|Tail]}, without
recurring to needed narrowing as in in
LazySmallCheck~\cite{smallcheck}.



Extending our tactic to handle computations (\texttt{simp}) and/or to
return the partial proof state, unsolved goals in particular. Mention
\emph{Harpoon}, Beluga's new tactic language?

Extending the LJF kernel to handle defined atoms

\todo{Link to the repo. Which one? \url{https://github.com/proofcert/fpc-elpi} is provate and full of stuff}
%%% Local Variables:
%%% mode: latex
%%% TeX-master: "main"
%%% End:

% LocalWords:  Elpi
