\section{Conclusion and future work}

This paper follows on a line of research  starting in the late 80 and gaining more steam in the last five years that advocates the usefulness of proof theory and higher-order logic programming  for the many tasks around the development, enrichment and even formal verification of proof assistants. The development of the Elpi-Coq plug-in  has made this connection tighter.

We have presented two application of this synergy: one geared towards
providing a flexible approach to connecting external provers of
first-order intuitionistic logic to Coq; the other supporting an out-of-the-box way to do property-based testing for pure inductive relations.
%

%\todo{sum up contribution}

There are many avenues of developments for this line of research.
The code reported in Fig.~\ref{fig:augmented} is somewhat idealized in
so far that we omitted a rule performing weak-head reduction (allowing
to handle directly existential goals) and more importantly one that uses
 Elpi's \emph{delay} mechanism. % This a distinguished feature of Elpi, whose
% usef has been advocated in ~\cite{DunchevCT16,coen19mscs}.
The
use of constraints for data generation is well
developed~\cite{FioravantiPS15} and we could try to leverage it to
improve our PBT tactic so that it can use partially instantiated terms: for example generating
say lists with at least two elements as \texttt{[1,2|Tail]}, without
recurring to needed narrowing as in 
LazySmallCheck~\cite{smallcheck}. On the more practical side, it would be worthwhile to investigate \emph{random} generation, following the ideas in~\cite{pltredexconstraintlogic,blanco19ppdp}.

Another obvious limitation of our PBT tactic is that it does not 
 handle any computation (conversion) within an inductive relation. We plan to tackle this either at the Ltac level by co-routining it with simplification tactics and/or by adding rewriting features to our FPC system~\cite{ChihaniM16}.
% Mention \emph{Harpoon}, Beluga's new tactic language?

 Finally it makes sense to tie together the two threads of this paper
 and provide a way of checking and elaborating proof evidence for
 intuitionistic logic \emph{with} (inductively) defined atoms. This
 can go as far as FPC for inductive proofs~\cite{blanco15wof}.

 \smallskip
 The source files mentioned in this paper are available at \url{https://github.com/proofcert/fpc-elpi}.
 % \begin{metanote}
%   Decided against a related work section: most is already discussed in the intro, the rest can be mentioned in passing --am
% \end{metanote}
% \subsection{Related work}
% \label{ssec:rel}

% \begin{itemize}
% \item Brief discussion/comparison with \texttt{auto} [done]
% \item Mention other papers by Enrico as applications of coq-elpi~\cite{coen19mscs,tassi19itp}\dots [done]
% \item Say we \textbf{do} not talk about Coq metaprogramming (Ltac, Mtac), do say that doing our to application in MetaCoq~\cite{sozeau2020metacoq} would be a disaster (reimplement unfication, backtracking, binders) [done]
% \end{itemize}

% \subsection{Future work}
% \label{ssec:fut}
%\todo{finish future work}

 % \todo{Link to the repo. Use \url{https://github.com/proofcert/fpc-elpi}, make it public andmove papers to parsifal}
 
%%% Local Variables:
%%% mode: latex
%%% TeX-master: "main"
%%% End:

% LocalWords:  Elpi simp LJF PBT LazySmallCheck Ltac
