%% Important dates:
%%  Paper submission: 19 October 31 October 2020
%%  Author notification: 18 January 2021
%%  Final version:  15 February 2021
%%  Publication (presumably): end of March 2021

%% Style requirements of LIPIcs.  The recommended length of a paper is
%% 12-15 pages, excluding front-page(s) (authors, affiliation, keywords,
%% abstract, ...), bibliography and an appendix of max 5 pages. If you
%% need more pages, please ask the editors.

%% Papers have to be submitted in pdf through EasyChair:
%% https://easychair.org/conferences/?conf=types2020postproceed 

\documentclass[a4paper,USenglish,cleveref, autoref, thm-restate]{lipics-v2019}
\input head
\input preamble

\title{Interfacing logic programming and Coq} % Other suggestions?
\titlerunning{Interfacing logic programming and Coq}

\begin{document}
\maketitle

\begin{abstract}
We do stuff
\end{abstract}

\section{Introduction}
\label{sec:intro}

Recently, C. Sacerdoti Coen and E. Tassi developed the ELPI
implementation \cite{dunchev15lpar} of \lP \cite{miller12proghol}.
More recently, Tassi has made ELPI available as a plugin to the Coq
proof assistant \cite{tassi18coqpl}.  This implementation of \lP
extends earlier implementations in primarily two directions.  First,
ELPI adds a notion of constraints, which makes it more expressive than,
say, the Teyjus implementation of \lP \cite{nadathur99cade}.  Second,
the plugin version of ELPI comes equip with an API for accessing the
Coq environment and a quotation and anti-quotation syntax allowing for
Coq expressions to be mixed in with \lP program elements.

ELPI provides a means for treating bindings and unification (of
structures containing bindings) that are convenient for various
meta-programming tasks in logic programming all things useful for
tactic programming in Coq (proof refinement).  For examples, ELPI
programmers are spared from dealing with low-level aspects of
representing bindings (e.g., De Bruijn indexes) while still having
clean and effective ways to manipulate binding structures.  In short,
ELPI appears to be a useful meta-language for Coq
\cite{coen19mscs,tassi18coqpl,tassi19itp}.  
Type inference is a well-known example of a meta-programming task for
which logic programming often provides immediate and elegant
implementations via the sound implementation of logic.  % In this paper,
% we present two additional project in which an implementation of proof
% search provides direct, elegant, and compact implementations of
% meta-programming tasks.
In this paper, we outline
two applications of the Coq plugin for ELPI that support this claim.
With these examples, we shall illustrate that ELPI is a useful
meta-language not just because for its treatment of bindings and
syntactic structures in general but also because it is actually based
on a sound implementation of a higher-order intuitionistic logic.
Before we present these two examples, we
first discuss the different world views that the integration of ELPI
and Coq forces us to confront.

\section{Two cultures}

In proof theory, one learns quickly that things always come in pairs:
negative/positive, left/right, bottom-up/top-down,
premises/conclusion, introduction/elimination,
% determinism/nondeterminism,
% invertible/noninvertible, 
etc.
%
When we examine the larger setting of this project of linking a logic
programming engine with Coq and its kernel, we find a large number of
pairing that are valuable to explicitly describe.

\subsection{Proof theory vs type theory}

In many ways, (structural) proof theory is more elementary and
low-level than most approaches to type theory.  For example, type
theories usually answer the question ``What is a proof?'' with the
response ``a dependently typed $\lambda$-term''.  That is, when
describing a type theory, one usually decides that a proof is a
certain kind of term within the system.  In contrast, the proof theory
treats logical propositions and proofs as separate.  For example,
proof theory does not assume that there are terms within the logic
that describe proofs.

\begin{metanote}
  This paragraph seem to identify proof and type theory with sequents and nat ded and while
  I agree with the gist (although people such as Frank would argue that sequents are a way of organizing proof search for nat ded), we're not gonna make a lot of friends at types and more importantly is a big hammer to nail the point that logic programming is helpful with Coq --am 
\end{metanote}

Gentzen's discovery that the key to treating classical and
intuitionistic logics in the same proof system was the invention of
the weakening and contraction structural rules on the right-side of
sequents~\cite{gentzen35}.  This discovery lead him away from natural
deduction to the invention of multiple-conclusion sequent calculus, a
very important kind of proof that has not been given a satisfactory
representation in type theory.  This same innovation of Gentzen also
opened the way to the invention of another key proof theoretic
discovery, that of linear logic \cite{girard87tcs}.
Similarly, sequent calculus provides an elegant presentation for
linear logic while most treatments of linear logic in type theory
usually only capture linearity in a partial fashion.

Proof theory also has rather immediate and natural treatments for
co-induction (as well as induction: see
\cite{baelde12tocl,momigliano12jal}) whereas the treatment of
coinduction within type theory remains more challenging and is well
developed only for various syntactic conditions (see, for example,
\cite{bertot08entcs}).
\begin{metanote}
  The criticism applies to Coq, less to type theory, see co-patterns --am
\end{metanote}

\subsection{Proof search vs proof normalization}

Gentzen-style proofs are used to model computation in at least two
different ways.  The functional programming paradigm, following the
Curry-Howard correspondence, models computation abstractly as the
$\beta$-reduction of natural deduction proof: that is, computation is
modeled using \emph{proof normalization}.  On the other hand, the
logic programming paradigm, following the notion of goal-directed
proof search, models computation abstractly as a regimented search of
cut-free sequent calculus proofs: that is, computation is modeled
using \emph{proof search}.

Proof search has proved to have some richness that is hard to capture
in the proof normalization setting.  In particular, Gentzen's
invention of \emph{eigenvariables} are a kind of a proof-level
binder.  In the proof normalization setting, such eigenvariables
are instantiated during proof normalization.  In contrast, during the
search for cut-free proofs, eigenvariables are part of the syntax of
terms and formulas.  As a result, they can be used in the treatment of
bindings in data structures more generally.  Hence, the logic
programming paradigm provides a completely natural, elegant, and
powerful method, called $\lambda$-tree syntax, for treating bindings
within syntax.

In this paper, we shall use two different proof systems.  The first
logic is first-order intuitionistic logic.  Here, predicates
are not inductively defined and quantification will be limited to
range over first-order term.  Actually, we shall examine only the Horn
clause fragment of that logic.
%
The second logic will allow for inductively defined predicates using a
fixed point notation.   Given again our restriction to essentially
Horn clause specifications, it is possible to view intuitionistic
reasoning with such specification as if they are actually
specification in $\mu$MALL, an extension of multiplicative additive
linear logic with the least and greatest fixed point operators
\cite{baelde12tocl,baelde07lpar,heath19jar}.
%
The difference between these two logics and associate proof theories
is reminiscent to the difference between the type \lsti{Prop}
(corresponding to undefined propositions) and Sets (corresponding to
inductively defined propositions).
\begin{metanote}
  Still not onboard with this --am
\end{metanote}

\subsection{\lP vs Coq}

Given that \lP and Coq are both examples of mixing the
$\lambda$-calculus with logic, it is important to understand their
differences.  The confusion around the term \emph{higher-order
abstract syntax} (HOAS), for example, helps to illustrate their
differences.  In the functional programming setting, including the Coq
system, the HOAS approach leads to encoding binding structures within
terms using functions.  Such encodings have results in surprisingly
complicated encodings, allowing, for example, for \emph{exotic terms}
\cite{despeyroux95tlca} and to structures on which induction is not
possible \cite{roeckl01fossacs}.  In the logic programming setting,
particularly \lP, HOAS is well supported in both the availability of
binders allowed with terms ($\lambda$-bindings), in formulas
(quantifiers), and in proofs (eigenvariables).  For this reason, the
term $\lambda$-tree syntax was introduced to name this particular take
on HOAS \cite{miller19jar}.  The Abella proof assistant
\cite{baelde14jfr} was designed, in part, to allow for inductive and
co-inductive proofs for specifications using the $\lambda$-tree syntax
approach.

Another difference between \lP and functional programming can be
illustrated by considering how they are used in the specification of
tactics that are popular in modern proof assistants.  In fact, the
origin story for the ML functional programming language was that ML
was the meta-language for implementing the LCF suite of tactics and
tacticals \cite{gordon79}.  In order to implement tactics, ML adopted
not only novel features such as polymorphically typed higher-order
functional programming but also the non-functional mechanisms of
failure and exception handling.  While \lP is also based on
polymorphically typed higher-order relational programming, it also
comes with a completely declarative version of failure and
backtracking search.  Combining those features along with its support
of unification (even in the presence of term-level bindings), \lP
provides a rather different approach to the specification of tactics
\cite{felty93jar}.

While \lP is a typed language, it only uses simple types and these are
more often used to indicate syntactic categories.  Any dependency
information needs to be captured using predicates and quantifiers. 


\input sec3

\input pbt

\input elab

\section{Conclusion}


\bibliography{l}
\end{document}

%%% Local Variables:
%%% mode: latex
%%% TeX-master: t
%%% End:

% LocalWords:  Tassi ELPI Teyjus intuitionistic Gentzen's Gentzen LCF
% LocalWords:  eigenvariables tacticals polymorphically
