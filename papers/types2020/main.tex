%% Important dates:
%%  Paper submission: 19 October 31 October 2020
%%  Author notification: 18 January 2021
%%  Final version:  15 February 2021
%%  Publication (presumably): end of March 2021

%% Style requirements of LIPIcs.  The recommended length of a paper is
%% 12-15 pages, excluding front-page(s) (authors, affiliation, keywords,
%% abstract, ...), bibliography and an appendix of max 5 pages. If you
%% need more pages, please ask the editors.

%% Papers have to be submitted in pdf through EasyChair:
%% https://easychair.org/conferences/?conf=types2020postproceed 

\documentclass[a4paper,USenglish,cleveref, autoref, thm-restate]{lipics-v2019}
\input head
\input preamble

\title{Two applications of logic programming to Coq} % Other suggestions?
% bit too general -- elpi-coq is such an interface -a,
% on the usefulness of logic programming in Coq
% two applications of logic programming to Coq
\titlerunning{Two applications of logic programming to Coq}

\begin{document}
\maketitle

\begin{abstract}
The \emph{foundational proof certificate} (FPC) framework is capable
of checking and elaborating a wide range of proof evidence and can easily
be implemented using the logic programming language \lP.
We use the recently released Coq-Elpi plugin for Coq in which the Elpi
implementation of \lP can send and retrieve information to and from the Coq kernel.
We provide two examples: first we implement a FPC-driven sequent calculus for a fragment of the Calculus of Inductive Constructions and we package it into a tactic to perform property-based testing of Inductive types corresponding to pure (higher-order) Horn programs. Secondly, we implement a proof checker for first-order LJF to demonstrate how  proof certificates
can be supplied by external (to Coq) provers and elaborated into the
fully detailed proof terms that can be checked by the Coq kernel.
\end{abstract}
 
\input intro
\input cultures
\input sec3
\input pbt
\input elab

\input conc
\bibliography{l}
%\newpage
\appendix
\input app
\end{document}

%%% Local Variables:
%%% mode: latex
%%% TeX-master: t
%%% End:

% LocalWords:  pbt elab conc Elpi LJF
