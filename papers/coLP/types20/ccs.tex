\RequirePackage[utf8]{inputenc}
\RequirePackage[T1]{fontenc}
\documentclass[a4paper]{easychair}
\usepackage{url,xcolor}
\usepackage[xcolor]{changebar}
\usepackage{listings}
\usepackage{letltxmacro}
%%%%%%%%%%%%%%%%%%%%%%%%%%%%%%%%%%%%%%%%%%%%%%%%%%%%%%%%%%%%%%%%%%%%%%%%%%%%%%%%
%%  Listings

\colorlet{lprolog}{blue}
% \colorlet{abellatop}{blue!70!green}
% \colorlet{abellatac}{orange!30!black}
% \colorlet{abellabad}{red!80!yellow}
%\definecolor{bleu}{HTML}{000DB3}

\lstdefinelanguage{lprolog}{%
  alsoletter={-},
  classoffset=0,%
  morekeywords={sig,module,type,kind,pi,sigma,end,infixl,infixr,o},%
  keywordstyle=\color{lprolog},%
  classoffset=0,%
  otherkeywords={:-,=>,<=,\&},%
  sensitive=true,%
  morestring=[bd]",%
  morecomment=[l]\%,%
  morecomment=[n]{/*}{*/},%
}

\lstdefinelanguage{abella}[]{lprolog}{%
  alsoletter={-},
  classoffset=1,%
  morekeywords={Close,CoDefine,Define,Kind,Query,Quit,Specification,
    Set,Split,Theorem,Type,Undo,by,as,prop,true,false,forall,exists},%
  keywordstyle=\color{blue},%
  classoffset=2,%
  morekeywords={abbrev,apply,backchain,case,coinduction,cut,
    induction,inst,intros,monotone,on,permute,rename,left,right,witness,
    search,split,to,unabbrev,unfold,assert,with},%
  keywordstyle=\color{abellatac},%
  classoffset=3,%
  morekeywords={undo,abort,skip,clear},%
  keywordstyle=\color{abellabad}\underbar,%
  classoffset=0,%
}

\lstdefinelanguage{mlts}[Objective]{Caml}{%
  morekeywords={nab, new},%
  otherkeywords={@,\\, =>},%
}

\lstset{%
%  basicstyle=\smaller\ttfamily,%
  basicstyle=\ttfamily,%
  breakatwhitespace=true,breaklines=true,%
  language=abella,%
  commentstyle=\itshape,%
  xleftmargin=5pt,
  rangeprefix=/*\ ,
  rangesuffix=\ */,
  includerangemarker=false,
}

\newcommand*{\SavedLstInline}{}
\LetLtxMacro\SavedLstInline\lstinline
\DeclareRobustCommand*{\lstinline}{%
  \ifmmode
    \let\SavedBGroup\bgroup
    \def\bgroup{%
      \let\bgroup\SavedBGroup
      \hbox\bgroup
    }%
  \fi
  \SavedLstInline
}
\DeclareRobustCommand\lsti[1][]{\lstinline[basicstyle=\ttfamily,keepspaces=true,#1]}

%%%%%%%%%%%%%%%%%%%%%%%%%%%%%%%%%%%%%%%%%%%%%%%%%%%%%%%%%%%%%%%%%%%%%%%%%%%%%%%%

  % Use to get better highlighing of code
\lstset{language=lprolog}
\lstset{language=abella}

\title{Notes on PBT and CCS}
\authorrunning{Momigliano}
\titlerunning{Notes on PBT and CCS}
\author{Alberto Momigliano}

\institute{
DI, Universit\`a degli Studi di Milano, Italy
}
\begin{document}

\maketitle


While PBT originated in the functional programming community, we have
given in a previous paper (\cite{Blanco0M19}) a reconstruction of some
of its features (operational semantics, different flavors of
generation, shrinking) in purely proof-theoretic terms employing the
framework of \emph{Foundational Proof Certificates}~\cite{chihani17jar}.
%
In the context of PBT, the FPC setup is quite simple.
Consider an attempt to find counter-examples to a conjecture of the
form \(\forall x [(\tau(x)\wedge P(x)) \supset Q(x)]\) where $\tau$ is
a typing predicate and $P$ and $Q$ are two other predicates defined
using Horn clause specifications.
%
By negating this conjecture, we attempt to find a (focused\footnote{
For Horn specs, focusing is immaterial, but when we do proof search over logic with fixed points, it helps to structure the search itself. Still, the rules are heavy-handed and we can just work with base sequents for the time being}) proof of 
\(\exists x [(\tau(x)\land P(x)) \land \neg Q(x)]\).
%
In the focused proof setting, the \emph{positive phase} (where
test cases are generated) is represented by \(\exists x\) and
\((\tau(x)\land P(x))\). 
%
That phase is followed by the \emph{negative phase} (where conjectured
counter-examples are tested) and is represented by \(\neg Q(x)\).
%
%% turns out to be a search for a proof of \(\exists x [(\tau(x)\land
%%   P(x)) \land \neg Q(x)]\) in a focused sequent calculus for
%% (basically) Horn logic, and pre-condition phase, while proving $Q$ can
%% be relegated to deterministic logic programming-like computation (the
%% \emph{negative} phase), interpreting ``$\neg$'' as
%% Negation-as-failure..
%

As explained in~\cite{Blanco0M19}, the standard PBT setup needs little
more than Horn logic. % specifications.
%
However, when addressing infinite computations, we need richer
specifications.  While coinductive logic programming,
see~\cite{Luke07} may at first seem to fit the bill, the need to model
infinite \emph{behavior} rather than infinite objects\footnote{I find
  the combination of Logic Programming and infinite terms difficult to handle (but I need to read~\cite{BasoldKL19}), so I suggest to restrict to finite objects that have meaningful infinite behaviors}, that is any infinite
terms on the domain of discourse, has lead us to adopt a much stronger
logic (and associated proof theory) with explicit rules for induction
and coinduction.

A natural choice for such a logic is the fixed point logic $\cal
G$~\cite{Gacek2012} and its linear logic cousin $\mu\mathrm{MALL}$~\cite{Baelde12}, which are associated to the \emph{Abella} proof
assistant (\url{http://abella-prover.org}) and the \emph{Bedwyr} model-checker (\url{http://slimmer.gforge.inria.fr/bedwyr/}).  In fact, the latter
has already been used for related aims~\cite{HeathM15}.

In our current setup, we attempt to find counter-examples, using
Bedwyr to execute both the generation of test cases  and the testing phase.
%
Such an implementation of PBT has the advantages of allowing us to
piggyback on Bedwyr's facilities for efficient proof 
search via tabling for (co)inductive predicates.

%and supports reasoning over lambda-tree syntax via the $\nabla$ quantifier. 


To make things more concrete, consider CCS. While analogous goals
using, for example, the \emph{Concurrency Workbench}, it is a
remarkable feature of the proof-theoretic account that is easy to
generalizes PBT from a system without bindings (say, CCS) to a system
with bindings (say, the $\pi$-calculus).
%

%
Now, for the syntax of CCS:
%\begin{changebar}
  

  Wrt a standard presentation, we use \texttt{null} for $0$,
  \texttt{plus} for binary sums, \texttt{par} for parallel
  composition, \texttt{out} for the prefix process. We do not account
  for relabeling or restriction (but it's easy with an operator
  \texttt{nu : (act -> proc) -> proc}). Recursion (aka process
  definitions) could be easily accounted for with a constant \texttt{mu
    : (proc -> proc ) -> proc}, but this has some issues when defining
  a generator for it, so we postpone it. Basically, the generator
  clause would be \verb|is_proc (s N) (mu P) := is_proc N (P (mu P))|
  and this is out of the HO pattern unification fragment that Bedwyr
  handles. We can fix that, making the substitution explicit, but I
  just have not done it yet.
%  \end{changebar}


\begin{lstlisting}
Kind    proc, act,num  type.
Type    a,  b, tau     act .
Type    bar            act -> act .
Type    null           proc.
Type    out            act -> proc -> proc.
Type    plus, par      proc -> proc -> proc.
\end{lstlisting}

  The transition system:
  \begin{changebar}
    Co-names are obtained via \texttt{bar} and the \texttt{comp}
    predicates mediates among them. \texttt{step} is just the
    translation of the operational semantics rules of the above
    fragment of CCS (no value-passing, restriction, and recursion).
  \end{changebar}
\begin{lstlisting}
Define comp : act -> act -> prop by
comp A (bar A);
comp (bar A) A.

Define step : proc -> act -> proc -> prop by
step (out A P) A P;
step (plus P _) A P1 := step P A P1;
step (plus _ Q) A Q1 := step Q A Q1;
step (par P Q) A (par P1 Q) := step P A P1;
step (par P Q) A (par P Q1) := step Q A Q1;
step (par P Q) tau (par P1 Q1) :=
  exists A B, comp A B /\ step P A P1 /\ step Q A Q1.
\end{lstlisting}

  Now for \emph{strong} (bi)similarity (coinductive) and its negation,
  which we obtain by pushing negation inside. This is informally
  justified by the fact that Bedwyr's proof theory can be seen as
  multiplicative additive linear logic with fixed points, where
  negation normal forms exist in the same sense of classical logic\footnote{Th story is a bit more complicated since to handle open queries, we also need disunification, but again this is relevant only wrt the current implementation of Bedwyr.}:
\begin{lstlisting}
Define coinductive sim : proc -> proc -> prop by
sim P Q :=
	forall A P1,  step P A P1  -> exists Q1,  step Q A Q1  /\ sim P1 Q1.

Define coinductive bisim : proc -> proc -> prop by
bisim P Q :=
    (forall A P1, step P A P1 -> exists Q1, step Q A Q1 /\ bisim P1 Q1) /\
    (forall A Q1, step Q A Q1 -> exists P1, step P A P1 /\ bisim P1 Q1).

Define inductive n_bisim : proc -> proc -> prop by
  n_bisim P Q :=
	(exists A P1, step P A P1 /\ forall Q1, step Q A Q1 -> n_bisim P1 Q1) \/
	(exists A Q1, step Q A Q1 /\ forall P1, step P A P1 -> n_bisim Q1 P1).
\end{lstlisting}

Generators: these are here hard-wired exhaustive depth-bounded
ones, but using the FPC architecture of~\cite{Blanco0M19}, we can mix
and match different strategies (even random ones); for the sake of
these notes, these generators will do.
\begin{lstlisting}
% indices for gens
Type z num.
Type s num -> num.

Define is_act : num -> act -> prop by
is_act  _ a ; is_act _ b ; is_act _ tau;
is_act (s N) (bar A) := is_act N A. 

Define is_proc :          num -> proc  -> prop by
is_proc _ null;
is_proc (s N) (out A P) := is_act N A /\ is_proc N P;
is_proc (s N) (plus P Q) := is_proc N P /\ is_proc N Q;
is_proc (s N) (par P Q) := is_proc N P /\ is_proc N Q.
\end{lstlisting}

Now, we can use directly Bedwyr, for example asking it to find two
terms that are similar but not bisimilar:
\begin{lstlisting}
?= N = (s (s z)) /\ is_proc N P /\ is_proc N Q /\ sim P Q /\ sim Q P /\ n_bisim P Q.
 % Q = par (out tau null) (out tau null),  that is t.0 | t.0
 % P = out tau (out tau null), that is t.t.0
\end{lstlisting}

The generation of a process is just logic programming search, while
testing their similarity is coinductive and it is implemented in
Bedwyr with tables, which basically encode the required
simulation. There is a meta-theoretic result that translates a tabled
logic programming derivation into one in a meta-logic such as $\cal G$
with explicit induction and coinduction rules.  Bedwyr carries out a
standard uniform proof of a goal such as \verb|P -> Q|, with one
exception: once \texttt{P} is assumed, we carry out a case analysis on
it by searching for a proof of it and apply the stream of computed
answer substitutions from \texttt{P} to the search of a proof of
\texttt{Q}. This case analysis has some technical limitations to make
it sound, which explains why we need to explicitly encode the negation
of bisimulation rather then simply call \verb|bisim P Q -> false|.

\begin{changebar}
  Another query: is there a process that is not bisimilar to the
  parallel composition with itself? In other terms, is this true:
  $\forall P, bisim\ P ( P | P)$? Yes, namely \texttt{a.0}.
\end{changebar}


\begin{lstlisting}
 ? N = (s z) /\ is_proc N P /\ n_bisim (par P P) P.
%  P = out a null
\end{lstlisting}

For a final example, consider \emph{trace} equivalence:
\begin{lstlisting}
Kind trace   type.
Type emp     trace.
Type tr      act -> trace -> trace.

Define trace : proc -> trace -> prop by
       trace _ emp ;
       trace P (tr A Trace) := step P A Q /\ trace Q Trace.

Define trace_equiv : proc -> proc -> prop by
       trace_equiv P Q := (forall Trace, trace P Trace -> trace Q Trace) /\
                          (forall Trace, trace Q Trace -> trace P Trace).
\end{lstlisting}
It's possible to separate trace equivalence and bi-similarity:
\begin{changebar}
\begin{lstlisting}
Define trace_eq_not_bisim : proc -> proc -> prop by
 trace_eq_not_bisim P Q :=
  N =  (s (s z)) /\ N1 =  (s (s (s z))) /\ is_proc N P /\ is_proc N1 Q
	 /\  trace_equiv P Q /\ n_bisim P Q.
% Q = plus (out a null) (out a (out a null))
% P = out a (out a null)
\end{lstlisting}
but this takes some 10 minutes, probably because we are building up two processes $P$ and $Q$
independently.
\end{changebar}
This is different from previous examples
in~\cite{Blanco0M19}, where generation was performed over judgments
such as typing (see preservation/progress). A possibility is to
generate $P$ and $Q$ together, following ideas in \cite{TNIQ}, whereby
we first generate $P$ and then obtain $Q$ by locally mutating the
former under some strong constraints. This is where
\emph{constrain-and-generate} could play a role somehow.
\begin{changebar}
  Note however that while \texttt{trace} has mode \texttt{+,-}, it is
  non-deterministic. Moreover, a predicate such as \texttt{trace\_equiv}
  has to be called with mode \texttt{+,+}, by the operational
  semantics of Bedwyr
\end{changebar}
Another avenue to investigate would be to \emph{pre-compute}
partitions of processes, possibly removing isomorphic ones (maybe
under the equational theory of process equivalence -- certainly
\texttt{P + Q} and \texttt{Q + P} ``are'' the same process). Then at
run time we do the testing bit. Papers by Tarau here may be relevant.

\begin{center}
  \begin{huge}
  \textbf{WARNING}
  \end{huge}
\end{center}

The previous examples DO NOT use coinduction, but follows only by case
analysis, that is would hold even if (bi)similarity is just \emph{a}
fixed point, not the greatest.

To find an example where we do use that, we move to the lambda-calculus:
 consider the usual rules for CBV
evaluation in the $\lambda$-calculus with constants, but define it
\emph{coinductively}, following see~\cite{2007-Leroy-Grall}:
\begin{lstlisting}
Type    zero, one    const.
Type    con          const -> tm.
Type    app          tm -> tm -> tm.
Type    fun          (tm -> tm) -> tm.

Define coinductive coeval: tm -> tm -> prop by
 coeval (con C) (con C);
 coeval (fun R) (fun R);
 coeval (app M N) V := 
   exists R W, coeval M (fun R) /\ coeval N W /\ coeval (R W) V.
% generators
Define wt : num -> env -> tm -> prop by
    wt _ _ (con zero) ;
    wt _ _ (con one) ;
    wt _ E (A)              := mem E A ;
    wt (s N) E (app X Y)    := wt N E X /\ wt N E Y ;
    wt (s N) E (fun F)      := nabla x, wt N (bind x E) (F (x)).

\end{lstlisting}
Is evaluation still deterministic?  And if not, can we find terms
\lstinline{M}, \lstinline{M1}, and \lstinline{M2} such that
\mbox{\lstinline{coeval M M1 /\\ coeval E M2 /\\ (M1 = M2 -> false)}}?
\begin{lstlisting}
 ?= wt (s (s (s z))) void M /\ wt z void M1 /\ wt z void M2 /\ coeval M M1 /\ coeval M M2 /\ (M1 = M2 -> false))

 Found a solution:  2195msa
 M2 = con one
 M1 = con zero
 M  = app (fun (x1\ app x1 x1)) (fun (x1\ app x1 x1))
\end{lstlisting}

Note that we need generators for \texttt{M1,M2}, since a coinductive
definition of evaluation requires mode \texttt{+,+}, and I do not
see how co-routining would help. The inductive version has mode
\texttt{+,-}, so we need to generate only \texttt{M}; in fact, we can
use \texttt{coeval M M1} \emph{directly} as a generator.


\bibliographystyle{abbrv} \bibliography{../colp,../../ppdp19/l}

\end{document}

%%% Local Variables:
%%% mode: latex
%%% TeX-master: t
%%% End:

%  LocalWords:  Coinductive QuickCheck utf inputenc fontenc Herbrand
%  LocalWords:  corecursion coinductive corecursive QuickChick CBV ir
%  LocalWords:  disjuncts PCFL LTS Harrop PBT Coq's coinductively CCS
%  LocalWords:  intuitionistic FPCs Bedwyr Bedwyr's proc finitary
%  LocalWords:  bisimulation bisim Tarau sequents disunification
%  LocalWords:  routining
