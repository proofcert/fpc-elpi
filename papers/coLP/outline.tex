\RequirePackage[utf8]{inputenc}
\RequirePackage[T1]{fontenc}
\documentclass[a4paper]{easychair}
\usepackage{url,xcolor}
\usepackage{proof}
\usepackage{listings}
\usepackage{letltxmacro}

\newcommand{\instan}[1]{\hbox{\sl grnd}~(#1)}
\newcommand{\Pscr}{{\mathcal P}}


\newcommand{\ok}{\checkmark}
% DM The use of pifont and txfonts broke typesetting for me: equality
% signs never appearred and some parentheses did not appear either.
%\newcommand{\noc}{\ding{56}}
\newcommand{\noc}{\ddag}
\newcommand{\lP}{$\lambda$Prolog\xspace}
\newcommand{\nat }{\hbox{\sl nat}\xspace}
\newcommand{\plus}{\hbox{\sl app}\xspace}
\newcommand{\lst}{\hbox{\sl lst}\xspace}
\newcommand{\sort}{\hbox{\sl sort}}
\newcommand{\rev}{\hbox{\sl rev}}
\newcommand{\ordered}{\hbox{\sl ordered}}

%\newcommand{\atac}{A2Tac\xspace}
\newcommand{\atac}{ACheck\xspace}
%\newcommand{\lra}{\mathrel{\longrightarrow}}
\newcommand{\lra}{\mathrel{\vdash}}
\newcommand{\seq}[2]{#1\lra #2}

\newcommand{\true}{tt}

%%%%%%%%%%%%%%%% LJF
\newcommand{\truen}{t^-}
\newcommand{\truep}{t^+}
\newcommand{\falsen}{f^-}
\newcommand{\falsep}{f^+}
\newcommand{\wedgep}{\wedge^{\!+}}
\newcommand{\wedgen}{\wedge^{\!-}}
\newcommand{\veep}{\vee^{\!+}}
\newcommand{\veen}{\vee^{\!-}}
\newcommand{\with}{\&}

\newcommand{\Nscr}{{\cal N}}
\newcommand{\Rscr}{{\cal R}}                                   % Used for an ambiguous rhs
%\newcommand{\Rscr}{\Delta_1\Downarrow\Delta_2}                % Used for an ambiguous rhs
\newcommand{\jUnf    }[4]{#1\mathbin\Uparrow#2\vdash#3\mathbin\Uparrow #4} % unfocused sequent
\newcommand{\jUnfG   }[2]{\jUnf{\Gamma}{#1}{#2}{{}}}           % unf sequ with \Gamma
\newcommand{\jUnfamb }[3]{#1\mathbin\Uparrow#2\vdash#3 \Rscr}  % unfocused sequent
\newcommand{\jUnfGamb}[1]{\Gamma\mathbin\Uparrow#1\vdash \Rscr}% unf sequ with \Gamma
\newcommand{\jLf     }[3]{#1\Downarrow#2\vdash#3}              % left focused sequent
\newcommand{\jLfG    }[1]{\jLf{\Gamma}{#1}{E}}                 % left foc seq with \Gamma 
\newcommand{\jRf     }[2]{#1\vdash #2\Downarrow}               % right focused sequent
\newcommand{\jRfG    }[1]{\jRf{\Gamma}{#1}}                    % right foc seq with \Gamma
%%%%%%%%%%%%%%%% 
\newcommand{\bxi}[1]{\blue{\Xi_{#1}}}
\newcommand{\bXi}[1]{\blue{\Xi_{#1} :\null}}

\newcommand{\andClerk}[3]{{\wedge_c}(#1,#2,#3)}
\newcommand{\falseClerk}[2]{f_c(#1,#2)}
\newcommand{\orClerk}[2]{{\vee_c}(#1,#2)}
\newcommand{\allClerk}[2]{\forall_c(#1,#2)}
\newcommand{\storeClerk}[3]{\hbox{\sl store}_c(#1,#2,#3)}

\newcommand{\trueExpert }[1]{{\true_e}(#1)}
\newcommand{\eqExpert }[1]{{=_e}(#1)}
\newcommand{\unfoldExpert}[2]{{\hbox{\sl unfold}_e}(#1,#2)}
\newcommand{\andExpert}[3]{{\wedge_e}(#1,#2,#3)}
\newcommand{\andExpertLJF}[6]{{\wedge_e}(#1,#2,#3,#4,#5,#6)}
\newcommand{\orExpert  }[3]{{\vee_e}(#1,#2,#3)}
\newcommand{\someExpert}[3]{\exists_e(#1,#2,#3)}
\newcommand{\initExpert}[2]{\hbox{\sl init}_e(#1,#2)}
\newcommand{\cutExpert}[4]{\hbox{\sl cut}_e(#1,#2,#3,#4)}
\newcommand{\decideExpert}[3]{\hbox{\sl decide}_e(#1,#2,#3)}
\newcommand{\releaseExpert}[2]{\hbox{\sl release}_e(#1,#2)}
%%%%%%%%%%%%%%%%

%
\newcommand{\step}{\longrightarrow}
\newcommand{\sstlc}{\textit{STLC}\xspace}
\newcommand{\vds}{\vdash_\Sigma}


\def\bnfas{\mathrel{::=}}
\def\bnfalt{\mid}

\def\lam{\lambda}
\def\Lam{\Lambda}
\def\arrow{\rightarrow}
\def\oftp{\mathord{:}}
\def\hastype{\mathrel{:}}
%%%%%%%%%%%%%%%%%%%%%%%%%%%%%%%%%%%%%%%%%%%%%%%%%%%%%%%%%%%%%%%%%%%%%%
% Editorials
%%%%%%%%%%%%%%%%%%%%%%%%%%%%%%%%%%%%%%%%%%%%%%%%%%%%%%%%%%%%%%%%%%%%%%

\long\def\ednote#1{\footnote{[{\it #1\/}]}\message{ednote!}}
%\long\def\ednote#1{\begin{quote}[{\it #1\/}]\end{quote}\message{note!}}
\newenvironment{metanote}{\begin{quote}\message{note!}[\begingroup\it}%
                         {\endgroup]\end{quote}}
\long\def\ignore#1{}

\newcommand{\todo}[1]{\begin{metanote}TODO: #1\end{metanote}}
%
%\setlength{\textwidth}{13.2cm}
%\setlength{\textheight}{19.9cm}


%%%%%%%%%%%%%%%%%%%%%%%%%%%




%%%%%%%%%%%%%%%%%%%%%%%%%%%%%%%%%%%%%%%%%%%%%%%%%%%%%%%%%%%%%%%%%%%%%%%%%%%%%%%%
%%  Listings

\colorlet{lprolog}{blue}
% \colorlet{abellatop}{blue!70!green}
% \colorlet{abellatac}{orange!30!black}
% \colorlet{abellabad}{red!80!yellow}
%\definecolor{bleu}{HTML}{000DB3}

\lstdefinelanguage{lprolog}{%
  alsoletter={-},
  classoffset=0,%
  morekeywords={sig,module,type,kind,pi,sigma,end,infixl,infixr,o},%
  keywordstyle=\color{lprolog},%
  classoffset=0,%
  otherkeywords={:-,=>,<=,\&},%
  sensitive=true,%
  morestring=[bd]",%
  morecomment=[l]\%,%
  morecomment=[n]{/*}{*/},%
}

\lstdefinelanguage{abella}[]{lprolog}{%
  alsoletter={-},
  classoffset=1,%
  morekeywords={Close,CoDefine,Define,Kind,Query,Quit,Specification,
    Set,Split,Theorem,Type,Undo,by,as,prop,true,false,forall,exists},%
  keywordstyle=\color{blue},%
  classoffset=2,%
  morekeywords={abbrev,apply,backchain,case,coinduction,cut,
    induction,inst,intros,monotone,on,permute,rename,left,right,witness,
    search,split,to,unabbrev,unfold,assert,with},%
  keywordstyle=\color{abellatac},%
  classoffset=3,%
  morekeywords={undo,abort,skip,clear},%
  keywordstyle=\color{abellabad}\underbar,%
  classoffset=0,%
}

\lstdefinelanguage{mlts}[Objective]{Caml}{%
  morekeywords={nab, new},%
  otherkeywords={@,\\, =>},%
}

\lstset{%
%  basicstyle=\smaller\ttfamily,%
  basicstyle=\ttfamily,%
  breakatwhitespace=true,breaklines=true,%
  language=abella,%
  commentstyle=\itshape,%
  xleftmargin=5pt,
  rangeprefix=/*\ ,
  rangesuffix=\ */,
  includerangemarker=false,
}

\newcommand*{\SavedLstInline}{}
\LetLtxMacro\SavedLstInline\lstinline
\DeclareRobustCommand*{\lstinline}{%
  \ifmmode
    \let\SavedBGroup\bgroup
    \def\bgroup{%
      \let\bgroup\SavedBGroup
      \hbox\bgroup
    }%
  \fi
  \SavedLstInline
}
\DeclareRobustCommand\lsti[1][]{\lstinline[basicstyle=\ttfamily,keepspaces=true,#1]}

%%%%%%%%%%%%%%%%%%%%%%%%%%%%%%%%%%%%%%%%%%%%%%%%%%%%%%%%%%%%%%%%%%%%%%%%%%%%%%%%

  % Use to get better highlighing of code
\lstset{language=lprolog}
\lstset{language=abella}
\title{On the Proof Theory of Property-Based Testing of Coinductive Specifications, or:\\
 PBT to Infinity and beyond}
\authorrunning{Blanco, Miller and Momigliano}
\titlerunning{PBT to Infinity and
   beyond}
\author{Roberto Blanco\inst{1} \and Dale Miller\inst{2} \and Alberto Momigliano\inst{3}}

\institute{
  INRIA Paris, France
  \and
  INRIA Saclay \& LIX, \'Ecole Polytechnique, France\\
\and
DI, Universit\`a degli Studi di Milano, Italy
}
\begin{document}

\maketitle

\section{Introduction}
\label{sec:intro}

\begin{center}
  [For the moment taken from the TYPES abstract, to be revised]
\end{center}

Reasoning about infinite computations via coinduction and corecursion
has an ever increasing relevance in formal methods and, in particular,
in the semantics of programming languages, starting
from~\cite{milner91tcs}; see also~\cite{2007-Leroy-Grall} for a
compelling example --- and, of course, coinduction underlies (the
meta-theory of) process calculi. This was acknowledged by researchers
in proof assistants, who promptly provided support for coinduction and
corecursion from the early 90's on, see~\cite{Paulson97,Gim95types}
for the beginning of the story concerning the most popular frameworks.

It also became apparent that tools that search for
refutations/counter-examples of conjectures prior to attempting a
formal proof are invaluable: this is particularly true in PL theory,
where proofs tend to be shallow but may have hundreds of cases.  One
such approach is \emph{property-based testing} (PBT), which employs
automatic test data generation to try and refute executable
specifications.  Pioneered by \emph{QuickCheck} for functional
programming~\cite{claessen00icfp}, it has now spread to most major
proof assistants~\cite{BlanchetteBN11,QChick}.

% DM I revised the following paragraph.  Did I get the sense right?

In general, PBT does not extend well to coinductive specifications
(an exception being Isabelle's \emph{Nitpick},
% \url{https://isabelle.in.tum.de/dist/doc/nitpick.pdf}
which is, however, a counter-model generator).  A particular
challenge, for example, for \emph{QuickChick} is extending it to work
with Coq's notion of coinductive via \emph{guarded} recursion (which
is generally seen to be an unsatisfactory approach to coinduction). We
are not aware of applications of PBT to other form of coinduction, such as \emph{co-patterns}~\cite{AbelPTS13}.

While PBT originated in the functional programming community, we have
given in a previous paper (\cite{Blanco0M19}) a reconstruction of some
of its features (operational semantics, different flavors of
generation, shrinking) in purely proof-theoretic terms employing the
framework of \emph{Foundational Proof Certificates}~\cite{chihani17jar}: the
latter, in its full generality, defines a range of proof structures
used in various theorem provers such as resolution refutations,
Herbrand disjuncts, tableaux, etc.
%
In the context of PBT, the proof theory setup is much simpler.

\begin{center}
  [etc\dots]
\end{center}


The contributions of this paper are
\begin{itemize}
\item Tighten the link of our previous reconstruction of PBT to the
  proof-theoretic approach of model checking in~\cite{heath19jar} by casting it in the setting of $\mu$MALL
\item Extend PBT beyond the Horn-nabla fragment to Harrop spec, first using only case analysis, then also co-induction
\item Experiment with a multi-level architecture, where FPC only takes
  care of generation and testing  is delegated to a system such as Bedwyr
\item Hopefully provide some examples (CCS, $\pi$ calculus (?), co-evaluation)
\item
  \begin{center}
[others \dots]    
  \end{center}

\end{itemize}

\section{The story so far}
\label{sec:tecap}

\begin{center}
  [Here we recap what we did in PPDP, in the uniform proof style, see fig~\ref{fig:augmented}: the connection with linear logic comes in the next section]
\end{center}

Consider an attempt to find counter-examples to a conjecture of the
form \(\forall x [(\tau(x)\wedge P(x)) \supset Q(x)]\) where $\tau$ is
a typing predicate and $P$ and $Q$ are two other predicates defined
using Horn clause specifications.
%
By negating this conjecture, we attempt to find a (focused) proof of 
\(\exists x [(\tau(x)\land P(x)) \land \neg Q(x)]\).
%
In the focused proof setting, the \emph{positive phase} (where
test cases are generated) is represented by \(\exists x\) and
\((\tau(x)\land P(x))\). 
%
That phase is followed by the \emph{negative phase} (where conjectured
counter-examples are tested) and is represented by \(\neg Q(x)\).
%
%% turns out to be a search for a proof of \(\exists x [(\tau(x)\land
%%   P(x)) \land \neg Q(x)]\) in a focused sequent calculus for
%% (basically) Horn logic, and pre-condition phase, while proving $Q$ can
%% be relegated to deterministic logic programming-like computation (the
%% \emph{negative} phase), interpreting ``$\neg$'' as
%% Negation-as-failure..
%
%

\begin{figure}
\newcommand{\XXi}{{\color{blue}{\Xi}}}
\[
\infer{\XXi\vdash G_1\wedge G_2}
      {\XXi_1\vdash G_1\qquad \XXi_2\vdash G_2\qquad \andExpert{\XXi}{\XXi_1}{\XXi_2}}
\qquad
\infer{\XXi\vdash \true}
      {\trueExpert{\XXi}}
\]
\vskip -6pt
\[
\infer{\XXi\vdash G_1\vee G_2}
      {\XXi'\vdash G_i\qquad \orExpert{\XXi}{\XXi'}{i}}
\qquad
\infer{\XXi\vdash \exists x. G}
      {\XXi'\vdash G[t/x]\qquad \someExpert{\XXi}{\XXi'}{t}}
\]
\vskip -6pt
\[
\infer{\XXi\vdash t = t}
      {\eqExpert{\XXi}}
\qquad
\infer{\XXi\vdash A}
      {\XXi'\vdash G \quad (A~\hbox{\tt :-}~G)\in\instan\Pscr
                     \quad \unfoldExpert{\XXi}{\XXi'}}
\]
\caption{A proof system augmented with proof certificates and
  additional ``expert'' premises.}
\label{fig:augmented}
\end{figure}


\section{Linear logic as the proof theory of model checking}
\label{sec:ll}



A natural choice is linear logic cousin $\mu\mathrm{MALL}$~\cite{Baelde12}, which is associated to  the \emph{Bedwyr} model-checker. %  In fact, the latter
% has already been used for related aims~\cite{HeathM15}.

\subsection{Horn programs as fixed points}
\label{ssec:lpfixed}

\begin{center}
  [Here we introduce fixed points, and re-state the rules in fig~\ref{fig:augmented} with only unfolding rules. We may also want to introduce our first coinductive examples, probably \texttt{coeval} (it's Horn). 
  ]
\end{center}

\subsection{Onwards to linear logic}
\label{ssec:ll}

\begin{center}
  [
  \begin{itemize}
  \item Here we introduce polarization, focusing. I assume we will be
    using only $\mu$F rather than all muMALL; and if so, perhaps we
    can simplify the syntax of the focusing judgment
  \item Not sure we want to put clercks and experts everywhere, if they play a part only during generation and that should stay in the Horn+nabla fragment
  \item Now we can introduce non Horn examples such as simulation and points out that the left unfolding rules do case analysis.
\end{itemize}
  ]
\end{center}

\subsection{Full system}
\label{ssec:fullmumall}

\begin{center}
  [Here we add the rules for (co)induction, init (and cut?). In the end we take the muFlogic from the JAR paper and the ``completeness'' results thereof]
\end{center}

\section{Implementation and case studies}
\label{sec:case}

\begin{center}
  [Here we discuss the ``architecture'', namely using the lightweight FPC for generators and Bedwyr for testing]
\end{center}  
\subsection{Examples}
\label{ssec:exs}

\begin{center}
  [Which esamples? The ones which work]
\end{center}

\subsection{Refinements}
\label{ssec:refs}

\begin{center}
  [Anything else to make this interesting: integration with
  proof-outlines, pre-computation of terms 'a la Tarau, full
  ``post-mortem'' FPC, that is also for the testing phase, to
  elaborate further the results, for examples for explanations etc., wild ideas about generating infinite terms with observation functions as in \emph{Hipster} \cite{EinarsdottirJP18}]
\end{center}

\section{Related work}
\label{sec:rel}

\begin{center}
  [to much details, just for my own sake, only a small portion will make it in the paper]
\end{center}
  While coinductive logic programming,
see~\cite{Luke07} and~\cite{BasoldKL19} for a much more principled and
in depth treatment, may at first seem to fit the bill, the need to
model infinite \emph{behavior} rather than infinite objects, that is (ir)rational terms on the
domain of discourse, has lead us to adopt a much stronger logic (and
associated proof theory) with explicit rules for induction and
coinduction.

\paragraph{Haskell}

QuickCheck can partially deal with infinite objects, provided that the properties themselves are computable. Lazy evaluation is no magic bullets: stream equality loops, but we can use the \emph{take} lemma to observe finite portions.

\paragraph{coLP}

Gupta and his students have been advocating (very, very incrementally)
for extending LP first only coinduction, then allowing both least and
greatest semantics~\cite{Luke07}. The approach is semantically based:
you start with the greatest complete Herbrand model, containing finite
and infinite (both rational and irrational) ground terms. Then they
modify the operational semantics dropping the occur check and relying
on cycle (same atom) and loop detection (same atom modulo
unification). They show soundness of the procedure wrt the greatest
complete Herbrand model. Completeness seems hopeless for
recursion-theoretic reasons as shown in~\cite{AnconaD15}.

\paragraph{Katya}
This has been much improved by Komendantskaya and friends (see~\cite{BasoldKL19}) \dots

Rather than relying on infinite terms via rational unification, they
introduce in the language of the logic a fixed point constructor
(which is then restricted to occur guarded, so as to be normalizing)
and its equational theory. While this may make sense for programming,
it seems problematic for a logical framework. Do we still have
adequacy of the encodings and canonical forms? Do \texttt{fix z.~cons
  0 z} and \texttt{fix z.~cons 0 (cons 0 z)} denote the same infinite
term or now the encodings refers to equivalence classes (perhaps wrt
bi-similarity) ? It would be hard to give a Curry-Howard reading of
this, since fixed points terms do not have a type classifying them and
the later modality does not have a proof term. Still, it could be
straighten up following~\cite{abelVezzosi:aplas14}. However, the
approach seems limited to guarded coinduction and may suffer from the
very problems raised by Ancona et al., namely  that a greatest fixed point semantics plus infinite (although) guarded terms gives unwanted results. For example, I think that  this is
still provable: \texttt{last (fix ones.~cons 1 ones) 2}. Finally, it's hard to implement in our setup, since we have to change the unification algorithm to take into account unfolding of fixed points (it's equi-recursive, in Peirce's terminology).

\paragraph{Terms Generation}

There's been considerable work on generation (mostly random) of
lambda-terms, with an emphasis on uniform
distributions~\cite{BendkowskiGT17}. Tarau generalizes it to arbitrary
first-order signatures~\cite{Tarau18} extending Remy's algorithm for
generating random binary trees \dots

\paragraph{Coaxioms}

In a lengthy list of papers such as ~\cite{AnconaDZ17, Dagnino19},
Ancona at al.\ argue against the use of gfixpoints as the semantics of
Aczel's style inference system. Starting from the desire to improve on
Leroy's coinductive operational semantics given its shortcomings, as
well as those of co-logic programming, they try to carve out undesired
results: take for example the logic program for last element of a list
and an infinite list of ones \texttt{Os = [1|Os]}. Then by a circular
proof you can prove \texttt{last Os 2}. We would argue that (1) we do
not care about infinite objects (2) in a guarded sense, the definition
of \texttt{last} is \emph{not productive} and it is rejected by Coq
(as a \textbf{Fixpoint} (but that proof holds with a
\textbf{Coinductive}) and by Agda's copatterns. They instead use
coaxioms (and then corules) to rule out those derivations. Those are
standard rules, but with a different interpretation: an atom is
provable if it has an infinite (even irrational ) proof, but whose
nodes have all a finite proof from the coaxioms only. Semantically
this corresponds to a fixed point that lies somewhere between least
and greatest. In the \texttt{last} example, the coaxiom would state
\texttt{last [X|XS] X}, which should force the element to be in the
list and rule out the offending derivation.  To me this notion of a
proof does not seem effective (how do I check the coaxiom condition
finitely?) and the coaxioms do not seem to have a declarative reading,
or, less charitably, that reading makes no sense. Still, they may have
more details in later papers. Essentially, it's an answer to a problem
I do not much care for.

\paragraph{Abella's current setup}

I think we all agree that infinite terms are an hassle, better to stay
clear from. Still, Abella's approach makes me worry: first, there is
no formal connection between the guarded version implemented and the
fixed point rules of the meta-logic, in particular compared to the strict correspondence exhibited by Beluga's script language
\emph{Harpoon}.  This can be fixed, I assume. While we cannot have a
``paradox'' such as the \texttt{last} example just because we have no
way to construct such a term, there are others mentioned by Ancona
(graph reachability) that are raised by finite terms. Moreover,
examples in the library such as \texttt{colist.thm} are bizarre,
since we prove properties of objects that we cannot construct. So, we fail to prove that
stream equality entails free equality,  but we cannot exhibit two streams that
refute the property.
%%% Local Variables:
%%% mode: latex
%%% TeX-master: "outline"
%%% End:

%  LocalWords:  Tarau Remy's Coaxioms Ancona gfixpoints Aczel's coLP
%  LocalWords:  coinductive Fixpoint Agda's coaxioms corules coaxiom
%  LocalWords:  Herbrand Katya Komendantskaya QuickCheck

\section{Conclusion}
\label{sec:conc}



\bibliographystyle{abbrv}
\bibliography{colp,../ppdp19/l}

\end{document}

%%% Local Variables:
%%% mode: latex
%%% TeX-master: t
%%% End:

%  LocalWords:  Coinductive QuickCheck utf inputenc fontenc Herbrand
%  LocalWords:  corecursion coinductive corecursive QuickChick CBV ir
%  LocalWords:  disjuncts PCFL LTS Harrop PBT Coq's coinductively CCS
%  LocalWords:  intuitionistic FPCs Bedwyr Bedwyr's nabla
