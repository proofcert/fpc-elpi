% Submissions must not exceed 12 pages ACM style 2-column (including
% figures, but excluding bibliography). 

%\documentclass[preprint]{sigplanconf}
\documentclass[sigconf]{acmart}


%\settopmatter{printfolios=false,printccs=false,printacmref}
\settopmatter{printfolios=false,printccs=true,printacmref}

\usepackage{xspace}
% The following two fonts broke the typesetting of some math environments.
%\usepackage{txfonts}
%\usepackage{pifont}
\usepackage{amssymb,amsmath}
\usepackage{proof}
%\usepackage{url,breakurl}
%\usepackage{booktabs}
%\usepackage{graphics} % use for the Pitts-Gabbay quantifier

\usepackage{listings}
\usepackage{letltxmacro}
%%%%%%%%%%%%%%%%%%%%%%%%%%%%%%%%%%%%%%%%%%%%%%%%%%%%%%%%%%%%%%%%%%%%%%%%%%%%%%%%
%%  Listings

\colorlet{lprolog}{blue}
% \colorlet{abellatop}{blue!70!green}
% \colorlet{abellatac}{orange!30!black}
% \colorlet{abellabad}{red!80!yellow}
%\definecolor{bleu}{HTML}{000DB3}

\lstdefinelanguage{lprolog}{%
  alsoletter={-},
  classoffset=0,%
  morekeywords={sig,module,type,kind,pi,sigma,end,infixl,infixr,o},%
  keywordstyle=\color{lprolog},%
  classoffset=0,%
  otherkeywords={:-,=>,<=,\&},%
  sensitive=true,%
  morestring=[bd]",%
  morecomment=[l]\%,%
  morecomment=[n]{/*}{*/},%
}

\lstdefinelanguage{abella}[]{lprolog}{%
  alsoletter={-},
  classoffset=1,%
  morekeywords={Close,CoDefine,Define,Kind,Query,Quit,Specification,
    Set,Split,Theorem,Type,Undo,by,as,prop,true,false,forall,exists},%
  keywordstyle=\color{blue},%
  classoffset=2,%
  morekeywords={abbrev,apply,backchain,case,coinduction,cut,
    induction,inst,intros,monotone,on,permute,rename,left,right,witness,
    search,split,to,unabbrev,unfold,assert,with},%
  keywordstyle=\color{abellatac},%
  classoffset=3,%
  morekeywords={undo,abort,skip,clear},%
  keywordstyle=\color{abellabad}\underbar,%
  classoffset=0,%
}

\lstdefinelanguage{mlts}[Objective]{Caml}{%
  morekeywords={nab, new},%
  otherkeywords={@,\\, =>},%
}

\lstset{%
%  basicstyle=\smaller\ttfamily,%
  basicstyle=\ttfamily,%
  breakatwhitespace=true,breaklines=true,%
  language=abella,%
  commentstyle=\itshape,%
  xleftmargin=5pt,
  rangeprefix=/*\ ,
  rangesuffix=\ */,
  includerangemarker=false,
}

\newcommand*{\SavedLstInline}{}
\LetLtxMacro\SavedLstInline\lstinline
\DeclareRobustCommand*{\lstinline}{%
  \ifmmode
    \let\SavedBGroup\bgroup
    \def\bgroup{%
      \let\bgroup\SavedBGroup
      \hbox\bgroup
    }%
  \fi
  \SavedLstInline
}
\DeclareRobustCommand\lsti[1][]{\lstinline[basicstyle=\ttfamily,keepspaces=true,#1]}

%%%%%%%%%%%%%%%%%%%%%%%%%%%%%%%%%%%%%%%%%%%%%%%%%%%%%%%%%%%%%%%%%%%%%%%%%%%%%%%%

  % Use to get better highlighing of code
\lstset{language=lprolog}
\lstset{language=abella}

% \newcommand{\blue}[1]{{\color[rgb]{0,0,1} #1}}
% \newcommand{\tup}[1]{\langle #1\rangle}
% \newcommand{\tupp}[2]{\blue{\langle #1,}#2{\blue{\rangle}}}


\newcommand{\instan}[1]{\hbox{\sl grnd}~(#1)}
\newcommand{\Pscr}{{\mathcal P}}


\newcommand{\ok}{\checkmark}
% DM The use of pifont and txfonts broke typesetting for me: equality
% signs never appearred and some parentheses did not appear either.
%\newcommand{\noc}{\ding{56}}
\newcommand{\noc}{\ddag}
\newcommand{\lP}{$\lambda$Prolog\xspace}
\newcommand{\nat }{\hbox{\sl nat}\xspace}
\newcommand{\plus}{\hbox{\sl app}\xspace}
\newcommand{\lst}{\hbox{\sl lst}\xspace}
\newcommand{\sort}{\hbox{\sl sort}}
\newcommand{\rev}{\hbox{\sl rev}}
\newcommand{\ordered}{\hbox{\sl ordered}}

%\newcommand{\atac}{A2Tac\xspace}
\newcommand{\atac}{ACheck\xspace}
%\newcommand{\lra}{\mathrel{\longrightarrow}}
\newcommand{\lra}{\mathrel{\vdash}}
\newcommand{\seq}[2]{#1\lra #2}

\newcommand{\true}{tt}

%%%%%%%%%%%%%%%% LJF
\newcommand{\truen}{t^-}
\newcommand{\truep}{t^+}
\newcommand{\falsen}{f^-}
\newcommand{\falsep}{f^+}
\newcommand{\wedgep}{\wedge^{\!+}}
\newcommand{\wedgen}{\wedge^{\!-}}
\newcommand{\veep}{\vee^{\!+}}
\newcommand{\veen}{\vee^{\!-}}
\newcommand{\with}{\&}

\newcommand{\Nscr}{{\cal N}}
\newcommand{\Rscr}{{\cal R}}                                   % Used for an ambiguous rhs
%\newcommand{\Rscr}{\Delta_1\Downarrow\Delta_2}                % Used for an ambiguous rhs
\newcommand{\jUnf    }[4]{#1\mathbin\Uparrow#2\vdash#3\mathbin\Uparrow #4} % unfocused sequent
\newcommand{\jUnfG   }[2]{\jUnf{\Gamma}{#1}{#2}{{}}}           % unf sequ with \Gamma
\newcommand{\jUnfamb }[3]{#1\mathbin\Uparrow#2\vdash#3 \Rscr}  % unfocused sequent
\newcommand{\jUnfGamb}[1]{\Gamma\mathbin\Uparrow#1\vdash \Rscr}% unf sequ with \Gamma
\newcommand{\jLf     }[3]{#1\Downarrow#2\vdash#3}              % left focused sequent
\newcommand{\jLfG    }[1]{\jLf{\Gamma}{#1}{E}}                 % left foc seq with \Gamma 
\newcommand{\jRf     }[2]{#1\vdash #2\Downarrow}               % right focused sequent
\newcommand{\jRfG    }[1]{\jRf{\Gamma}{#1}}                    % right foc seq with \Gamma
%%%%%%%%%%%%%%%% 
\newcommand{\bxi}[1]{\blue{\Xi_{#1}}}
\newcommand{\bXi}[1]{\blue{\Xi_{#1} :\null}}

\newcommand{\andClerk}[3]{{\wedge_c}(#1,#2,#3)}
\newcommand{\falseClerk}[2]{f_c(#1,#2)}
\newcommand{\orClerk}[2]{{\vee_c}(#1,#2)}
\newcommand{\allClerk}[2]{\forall_c(#1,#2)}
\newcommand{\storeClerk}[3]{\hbox{\sl store}_c(#1,#2,#3)}

\newcommand{\trueExpert }[1]{{\true_e}(#1)}
\newcommand{\eqExpert }[1]{{=_e}(#1)}
\newcommand{\unfoldExpert}[2]{{\hbox{\sl unfold}_e}(#1,#2)}
\newcommand{\andExpert}[3]{{\wedge_e}(#1,#2,#3)}
\newcommand{\andExpertLJF}[6]{{\wedge_e}(#1,#2,#3,#4,#5,#6)}
\newcommand{\orExpert  }[3]{{\vee_e}(#1,#2,#3)}
\newcommand{\someExpert}[3]{\exists_e(#1,#2,#3)}
\newcommand{\initExpert}[2]{\hbox{\sl init}_e(#1,#2)}
\newcommand{\cutExpert}[4]{\hbox{\sl cut}_e(#1,#2,#3,#4)}
\newcommand{\decideExpert}[3]{\hbox{\sl decide}_e(#1,#2,#3)}
\newcommand{\releaseExpert}[2]{\hbox{\sl release}_e(#1,#2)}
%%%%%%%%%%%%%%%%

%
\newcommand{\step}{\longrightarrow}
\newcommand{\sstlc}{\textit{STLC}\xspace}
\newcommand{\vds}{\vdash_\Sigma}


\def\bnfas{\mathrel{::=}}
\def\bnfalt{\mid}

\def\lam{\lambda}
\def\Lam{\Lambda}
\def\arrow{\rightarrow}
\def\oftp{\mathord{:}}
\def\hastype{\mathrel{:}}
%%%%%%%%%%%%%%%%%%%%%%%%%%%%%%%%%%%%%%%%%%%%%%%%%%%%%%%%%%%%%%%%%%%%%%
% Editorials
%%%%%%%%%%%%%%%%%%%%%%%%%%%%%%%%%%%%%%%%%%%%%%%%%%%%%%%%%%%%%%%%%%%%%%

\long\def\ednote#1{\footnote{[{\it #1\/}]}\message{ednote!}}
%\long\def\ednote#1{\begin{quote}[{\it #1\/}]\end{quote}\message{note!}}
\newenvironment{metanote}{\begin{quote}\message{note!}[\begingroup\it}%
                         {\endgroup]\end{quote}}
\long\def\ignore#1{}

\newcommand{\todo}[1]{\begin{metanote}TODO: #1\end{metanote}}
%
%\setlength{\textwidth}{13.2cm}
%\setlength{\textheight}{19.9cm}


%%%%%%%%%%%%%%%%%%%%%%%%%%%


\begin{document}
%\conferenceinfo{}{}
%\CopyrightYear{}
%\copyrightdata{}
%\title{Property-Based Testing via Proof Reconstruction: Work-in-progress}

%\acmConference[PPDP'2019]{21th International Symposium on Principles and Practice of Declarative Programming}{October 7--9 }{Porto, Portugal}

\title{Notes on coinductive PBT}
\author{Roberto Blanco}
\affiliation{\institution{INRIA Paris, France}}
\email{roberto.blanco@inria.fr}
\author{Dale Miller}
\affiliation{\institution{INRIA Saclay \& LIX, \'Ecole Polytechnique, France}}
\email{dale.miller@inria.fr}
\author{Alberto Momigliano}
\affiliation{\institution{DI, Universit\`a degli Studi di Milano, Italy}}
\email{momigliano@di.unimi.it}

\begin{abstract}
  The \emph{linear } logic behind Bedwyr (``muFa'', according to
  Quentin) seems to be the proof-theory of coinductive PBT. Applications are mostly in separating
  equivalences in process/lambda calculi.
\end{abstract}
\maketitle

\section{The proposal}
\label{sec:p}



Our starting point is to see whether the FPC approach in the previous
paper extends to richer set of specifications. Originally, we wanted
to prove formulas of the form
%
\(\exists x [(\tau(x)\land P(x)) \land \neg Q(x)]\)
%
whether $P,Q$ were essentially Horn logic programs plus nabla. We
argued that the logic behind it was basically minimal logic, since we
only provided certificates for the positive phase, while the
negative/testing one was just exhaustive search.

Now, we want to PBT specs over \emph{infinite behavior} (more than infinite
objects). Here's an example: it is well known that in CCS there are processes that are similar to each other but not bisimilar,
for example \texttt{a.b + a} and \texttt{a.b}:

\begin{lstlisting}
Theorem sim_not_bisim: exists P Q,
 sim P Q /\ sim Q P /\ (bisim P Q -> false).
\end{lstlisting}
My Abella proof uses reflexivity of similarity in the positive phase,
but there is also a direct coinductive proof with the following
simulation:
\begin{enumerate}
\item $R_1 = \{\mathtt{(a.b+a, a.b ),(b,b),(0,b),(0,0)}\}$
  \item $R_2 = \{\mathtt{(a.b,a.b+a),(b,b), (0,0)}\}$ 
\end{enumerate}
%

More examples: in \cite{2007-Leroy-Grall}, it is suggested to take the usual rules for CBV in the lambda-calculus with constants, but \emph{coinductively}:
\begin{lstlisting}
CoDefine coeval: tm -> tm -> prop by
 coeval (con C) (con C);
 coeval (fun R) (fun R);
 coeval (app M N) V := exists R W, coeval M (fun R) /\ coeval N W /\ coeval (R W) V.
\end{lstlisting}

Whether this notion of co-evaluation makes sense is debatable, but the
following holds: it is not deterministic anymore, since a divergent
term such as $\Omega$ co-evaluates to anything
\begin{lstlisting}
Theorem fail_det: exists E V1 V2, coeval E V1
/\ coeval E V2 /\ (V1 = V2 -> false).
\end{lstlisting}
The proof uses coinduction twice with the trivial simulation
$\Omega,0$, and $\Omega, 1$ say.
%

Consider now \emph{ground} vs.\ \emph{applicative} similarity in
PCFL. The latter is finer than the former:

\begin{lstlisting}
CoDefine sim : tm -> tm -> ty -> prop by
sim  M1 M2 (arr S S') :=
 (forall M, {eval M1 (abs M)} -> 
 exists M' , {eval M2 (abs M')} /\ 
 (forall x, {of x S} -> sim  (M x) (M' x) S'));  ...
CoDefine gsim : tm -> tm -> ty -> prop by
 gsim  M1 M2 (arr S S') :=
 (forall x, {of x S} -> gsim  (app M1 x) (app M2 x) S');  ...
Theorem cex_asim_gsim: exists M1 M2 T, gsim M1 M2 T /\ (sim M1 M2 T -> false).
\end{lstlisting}
Two such separating terms are: $\lambda x.\ \bot$ and $\bot$, which
are not applicative similar since $\bot$ does not converge to a
lambda. The proof uses several lemmas such as transitivity of
\texttt{asim}, Kleene order being included in similarity, $\bot$ being
the least element in the similarity order etc.

Our angle: Can PBT find those processes/lambda terms?
All of this begs for a richer logic:
\begin{itemize}
\item   \texttt{(bi)sim} are not Horn clauses
\item they require some form of coinduction for proof search to succeed (and case analysis for the negative phase)
\end{itemize}

The combination of those two points rules out falling back on
coinductive lp as in
\url{https://www.swi-prolog.org/pldoc/doc/_SWI_/library/coinduction.pl},
based on \cite{Luke07}. Even $\lambda$Prolog seems out as we need to
do case analysis on eigenvariables and more crucially we cannot
simulate nabla with pi for non-Horn specs such as similarity (of
course we would need tables, but that's doable). We need a logic with
(co)induction and nabla. Sounds familiar?

\subsection{Shall we use Bedwyr as is?}
\label{sec:bed}


That's the first instinct, since it has automation for search and case analysis, the correct distinction
between eigenvariables and scoped constants, tabling and even tabling modulo lemmas (up to version 1.3). And in fact,  we
can investigate co-evaluation immediately, with hand-made height generators:
\begin{lstlisting}
 Define coinductive  coeval : tm -> tm -> prop by ...
 Define wt : num -> env -> tm -> prop by ... % generator
 ?= wt (s (s (s z))) void M /\ wt z void M1 /\ wt z void M2 /\ coeval M M1 /\ coeval M M2 /\ (M1 = M2 -> false))

 Found a solution:  2195msa
 M2 = con one
 M1 = con zero
 M  = app (fun (x1\ app x1 x1)) (fun (x1\ app x1 x1))
\end{lstlisting}
Note that we need generators for \texttt{M1,M2} since standard mode
analysis guaranteeing groundness of the output variable does not
immediately apply to coinductive LP.

For the sim/not bisim example, checking \texttt{bisim P Q -> false}
runs afoul of the limitations of the left proved that does not handle
universal and implications.  However, as noted in \cite{HeathM15}, the
logic behind Bedwyr is essentially linear and therefore we can rely on
negation normal forms and simply push negation inside to define
\emph{non}-bisimulation, yielding an interesting polarization
structure:

\begin{lstlisting}
Define inductive n_bisim : proc -> proc -> prop by
 n_bisim P Q :=
 (exists A P1, step P A P1 /\ forall Q1, step Q A Q1 -> n_bisim P1 Q1) \/
 (exists A Q1, step Q A Q1 /\ forall P1, step P A P1 -> n_bisim Q1 P1).

 ?= N = (s (s z)) /\ is_proc N P /\ is_proc N Q /\ sim P Q /\ sim Q P /\ n_bisim P Q.
Found a solution: 301ms
 Q = par (out tau null) (out tau null)
 P = out tau (out tau null)
\end{lstlisting}

A similar problem: find a process that is not bisimilar to its \texttt{par}

\begin{lstlisting}
?=is_proc (s z)) P /\ n_bisim (par P P) P).
  P = out a null
\end{lstlisting}

Question:
\begin{itemize}
\item What's going on recursion-theoretically? The complement of a
  r.e.\ coinductive relation must be inductive, but what is its
  degree? Conversely, the complement of an r.e.\ inductive relation
  such as evaluation is coinductive, certainly not r.e., so what we
  would get by negation normal form would be badly incomplete.
\end{itemize}

To recover the flexibility of different generating strategy, we can
port the FPC approach of the PPDP paper to Bedwyr. Current attempt is
doing this \emph{only for generation} and rely on Bedwyr for everything
else. 
\begin{lstlisting}
?= check  (qheight 2) (isProc (p2i P)) /\ (n_bisim (par P P)  P).
 \end{lstlisting}
From first experiments this is feasible, so far and allows us to use native tabling and the
L0/L1 search strategy. It's less efficient that I was expecting: a query like the above works, but
\begin{lstlisting}
(check (qheight 4 (and (isProc (p2i P)) (isProc (p2i Q)))
	/\ sim P Q /\ sim Q P /\ (n_bisim P Q)
\end{lstlisting}
takes some 20 secs, because we are building up two processes $P$ and $Q$
independently. This is different from previous examples in PPDP, where
generation was performed over judgments such as typing (see
preservation/progress). A possibility is to generate $P$ and $Q$ together, following
ideas in \cite{TNIQ}, whereby we fist generate $P$ and then obtain $Q$ by locally mutating the former.

This architecture may be also  less flexible, since we have to
live with the limitation of the left prover. And we do not have
certificates for the rest of the query. This may be limiting if we
need to steer both the satisfaction of the premises and the testing
phase, which is not purely negative anymore. 

Or we can go all the way two-level and use a kernel for muFa, pretty
much as in Quentin's code
\url{http://slimmer.gforge.inria.fr/bedwyr/pcmc/}. It's going to be
slow, especially if we use it for generation as well and it uses
standard (co)induction rules via invariants, which is not exactly
user-friendly. We could add support for tables, but we run the risk of
re-inventing the wheel (read Bedwyr). However, it is as flexible as we
program it, in the sense that we can add tabling modulo (sort of
testing ``up-to'' previously established theorems).

As a refinement of the first approach, the logic of Bedwyr not being
completely clear, one could take the cex found by Bedwyr and re-run it
in Quentin's kernel to have an honest-to-goodness proof in $\mu$mall
(although we need to turn tables into invariants).

 
\subsection{HM logic}
\label{sec:hml}
In \cite{HeathM15}, certificates for non (bi)similarity in LTS are assertions
in a fragment of Hennessy-Milner logic that separates two
processes. This begs the question if it makes sense to formulate such
a logic for PCFL and simulation. In other terms, show two (lazy) programs not contextually equivalent by generating an HML formula that separates them.

The presentation in \cite{Gordon95}
where small step semantics is annotated with observations (essentially
destructors) seems particularly apt and corresponds to ground
bisimilarity --- it's possible that by changing the dichotomy between
passive/active types it could be applied to the applicative case as
well.  See also\cite{LagoR15} for a similar LTS for applicative.

First question: is HML enough or do we have to go to the modal $\nu$-calculus to accommodate infinite computations? Example: \texttt{iterate f x} does not simulate
\texttt{map f (iterate (f x))}. Let \texttt{f} be the square function. The programs have transitions (using Gordon's LTS):
\begin{lstlisting}
iterate f 3 >> 
  3 :: iterate (f (f 3)) --hd->  3 --3-> stuck 
  3 :: iterate (f (f 3)) --tl->  iterate (f (f 3)) --?? ->

map f (iterate (f 3)) >>
  f 3 :: map f (iterate f (f 3)) --hd-> 9 etc
  f 3 :: map f (iterate f (f 3)) --tl->  map f (iterate f (f 3))
\end{lstlisting}
so a formula satisfying the former but not the latter should be something like
\texttt{(<hd> <3> true)}. In the first case we observe here 3 and  9 in the other. This suggests that we
can extract a separating \emph{evaluation context} from a HM formula, here essentially \texttt{hd \_}.

Some problems:
\begin{itemize}
\item Differently from LTS (and from the equational theory of the lambda
  calculus), our operational semantics is deterministic in the sense
  that while we can make different observations on the same term, we
  cannot make different transitions with the same observation. Hence
  the modalities seems to collapse and I see box as only useful for saying that no observations can be made. But are they still dual?
\item If we use Gordon's active/passive approach, we need a satisfaction clause for \texttt{Trans  Red}:
  $$
  P \models F
  \mbox{ if for type of $P$ active, when } P\Downarrow V, V \models F
  $$
\item We may have to add fixed points to account for formulae satisfying bisimilar terms such as \texttt{ones} and \texttt{ones'}
\begin{lstlisting}
zeros = 0 :: zeros
ones  = 1 :: ones
ones' = map (fun x -> x + 1) zeros 
\end{lstlisting}
\end{itemize}

Other questions
\begin{itemize}
\item HM logic is presented as parameterized by
the underlying LTS, but can we also vary the motion of bisimulation:
strong vs weak, up-to? There is some generalization of LTS into synchronization trees that may do that.

\item  What are the relations with
trace semantics and related notions such as in
\cite{2007-Leroy-Grall}?

\item In a sense to be made precise could FPC
induce a related HM logic?
\end{itemize}
In the meantime, we can use PBT HM logic to find formul\ae\ that separates
two non-similar LTS, here the example in Section 5.2 of
\cite{HeathM15}, where \texttt{hsat} is HM forcing of a modal formula
by a process:
\begin{lstlisting}
?= N = (s (s (s z))) /\ is_hform N F /\ hsat (proc 6) F  /\ (hsat (proc 1) F -> false).

F = hdiam a (hconj (hdiam b tt) (hdiam c tt))
N = s (s (s z))
\end{lstlisting}
Note that we cannot directly use box, as the left prover
complains. However, we may get around it defining non-forcing.




\subsection{Other  examples}

Because of the immense body of work in model checking (finite) process
calculi such as CCS, if we want to sell our approach, we may want as
usual to insist on domains that are hard for weaker frameworks, so
stick to binders. This means the $\pi$ and $\lambda$ calculi.

From the meta-theory of $\pi$ calculus
\begin{itemize}
\item  early and late bisim not preserved by inputs
\item  ground bisim not a congruence
\item ground  bisim not preserved by name substitution
\item model checking with the HML described in \cite{Tiu:2010} . Note that for the
  encoding of satisfaction to be adequate we need to compute the list
  of possible free new names beside those occurring in the process and
  formula. Still `` One can thus calculate the number of new names
  needed based on the number of bound input modalities in A.'' (ibidem page 21). See also  some of the examples in the Bedwyr example suite \verb+{pi/pi_modal.def}+
\end{itemize}


 More from \cite{2007-Leroy-Grall}:
\begin{itemize}
  \item There are raw terms that diverge, but not coevaluate;
\begin{lstlisting}
?= wt (s (s (s (s z)))) void M  /\ divrg M /\ (forall V, coeval M V -> false).

Found a solution: + 2011
 M = app (con zero) (app (fun (x1\ app x1 x1)) (fun (x1\ app x1 x1)))
\end{lstlisting}
  \item Let \texttt{costeps} be the kleene closure of the step relation, coinductively. Then the are terms such that
    \texttt{costeps E V} but not \texttt{coeval E V} --- the reverse of Lemma 36.
  \item type soundness fails for co-evaluation (the Filinski cex, based on the $Y$ combinator, seems hard, see Section 8.2).
  \end{itemize}

  More from the lazy lambda-calculus:
  \begin{itemize}
  \item The eta  and surjective pairing laws hold for ground similarity but not for applicative.
  \end{itemize}

  
  Other PL examples that we could mutate and see what happens:
\begin{itemize}  

\item Milner-Tofte type preservation result for circular closures~\cite{milner91tcs}
\item Transitivity of subtyping in recursive types (seen as rational trees over simple types)
  \cite{kozen_silva_2017}
\end{itemize}

\subsection{Related work}

Some precursors to coinductive LP \cite{JaffarSV04,CharatonikMNPW98},
all that much ado about nothing following \cite{Luke07} such as
\cite{DBLP:journals/cl/Ancona13,MantadelisRM14}. Note: the latter
seems intertwined with infinite terms rather than infinite behaviors
and does not have enough abstractions to tackle the examples we care
about.

Gupta says it can prove two Buchi automata bisimlar if the accepts the
same infinite strings (probably with some trick to express universal
quantification). For non equivalence one can show that there's a
string that the other does not accept, but the status of NAF wrt coLP
is totally unsettled. Anyhow, the automata example does not work in
SWI (but it does with the meta-interpreter).

NOTE: the CWB
\url{https://homepages.inf.ed.ac.uk/perdita/cwb/summary.html} can do a
bunch of stuff such as ``derive automatically logical formulae which
distinguish nonequivalent'' processes (of course, it is finite-state
systems). Perhaps try the new versions \url{http://caal.cs.aau.dk/}
\cite{CAAL} or \url{https://sourceforge.net/projects/cwb-nc/}. Also
the usual XSB tabled approach to simulation checking~\cite{symbolic}. Thus, we need to stress the pi calculus angle (binders, infinite state)

TO READ: stuff from Katya Komendantskaya \cite{KomendantskayaL17,BasoldKL19} (probably not that relevant). Stuff about testing equivalences in LTS (Sangiogi Book 1)
\bibliographystyle{ACM-Reference-Format}
\bibliography{colp}

\end{document}
%%% Local Variables:
%%% mode: latex
%%% TeX-master: t
%%% End:

%  LocalWords:  coinductive Blanco INRIA Saclay Ecole Polytechnique
%  LocalWords:  Universit degli Studi di Milano nabla CBV PCFL asim
%  LocalWords:  Kleene lp eigenvariables bisim Milner Tofte Katya HML
%  LocalWords:  Komendantskaya colp muFa PPDP Bedwyr formul LTS hsat
%  LocalWords:  coevaluate costeps kleene Filinski cex surjective CWB
%  LocalWords:  XSB bisimulation bisimilarity hd
